\documentclass[english]{article}
\usepackage[T1]{fontenc}
\usepackage[latin9]{inputenc}
\usepackage{textcomp}
\usepackage{amsmath, amsfonts}
\usepackage{babel}
\usepackage{lscape}
\usepackage{hyperref}
\usepackage{todonotes}

\makeatletter
\@ifpackageloaded{tex4ht}{%
  \def\pgfsysdriver{pgfsys-tex4ht.def}%
}{%
  % only needed inside a class
  \begingroup\expandafter\expandafter\expandafter\endgroup
  \expandafter\ifx\csname HCode\endcsname\relax
  \else
    \def\pgfsysdriver{pgfsys-tex4ht.def}%
  \fi
}
\makeatother

\usepackage{tikz}

\usepackage{cite}
\usepackage{url}


\begin{document}

\author{K. Sjobak, J. Molson}
\title{Compiling, building, and testing SixTrack}
\date{Jan. 2017}

\maketitle

\begin{abstract}
  The SixTrack code supports a wide variety of compile time options,compilers, and build environments.
  This document intends to present what is supported, and how to build the most common versions.
  Furthermore, an important sub-task of building SixTrack is to test that the built binary is correct; this is also covered.
\end{abstract}

\tableofcontents
\newpage

\section{Downloading SixTrack}
The last stable release of SixTrack should be found on GitHub, in the SixTrack repository under the SixTrack organization:
\url{https://github.com/SixTrack/SixTrack}.
This provides the source code, which you'll then have to compile as described in Section~\ref{sec:building}.

In order to download a release, you can either clone the git repository from our GitHub page, or download a tarball.
Since installs downloaded as tarballs cannot be automatically updated, and there is no way to check if something has been modified after unpacking, \textbf{it is recommended to use git}.
If you still want to download a tarball, the newest version is found at\\
\url{https://github.com/SixTrack/SixTrack/archive/master.zip}, and older versions can be found through
\url{https://github.com/SixTrack/SixTrack/releases}.

Assuming that you have a \texttt{git} client installed, you can clone the repository anonymously using the command:\\
\texttt{git clone https://github.com/SixTrack/SixTrack.git}\\
You shall now get a new folder ``SixTrack'' within your current working directory.
This folder contains the files from the newest version of SixTrack's ``master'' branch.
Furthermore, it also contains a full clone of the original repository, meaning that you can check out old versions, create branches, commit changes, etc. completely off line.

Assuming that no files tracked by version control have been modified, updating your local copy is done by running the command
\texttt{git pull} anywhere in the repository.
If you want to modify SixTrack, get feedback on your changes, and possibly contribute your changes back to the master branch; please also see Section~\ref{sec:downloading:contributing}.

\subsection{Modifying sixtrack and contributing}
\label{sec:downloading:contributing}


forks and stuff.

Setting up git and github.

\section{Building SixTrack}
\label{sec:building}

The SixTrack source code is located in the ``SixTrack'' subdirectory of the ``SixTrack'' repository.
SixTrack is normally configured and built using CMake, and for simplicity a wrapper \texttt{cmake\_six} is provided.
This allows configuring the various build options, changing compilers, and changing build types.
To run it, simply execute the command:\\
\texttt{cmake\_six \textit{compiler} \textit{buildtype} \textit{OPTION1} \textit{OPTION2} \textit{-OPTION3}}\\
Here the \texttt{\textit{compiler}} argument specifies the compiler to use, and the \texttt{\textit{buildtype}} argument whether to build a \texttt{release} or \texttt{debug} version.
To take the default compiler and build type, simply leave these options out.

The other options (all UPPER CASE) switch on or off code features as described in Section~\ref{sec:building:options}.
To see the list of options and their meaning, simply run \texttt{cmake\_six help}.
Note that to switch an option off put a ``\texttt{-}'' sign in front of its name, i.e.\ \texttt{-OPTION}.

Each build will produce a new subfolder\\
{
  \texttt{SixTrack\_\-cmakesix\_\-OPTION1\_\-OPTION2\_\-NOOPTION3\_\-compiler\_\-buildtype}~,\\
}
and each of the folders will contain a binary named\\
{\scriptsize
  \texttt{SixTrack\_\textit{VERSION}\_feature1\_feature2\_feature3\_etc\_cmake\_\textit{COMPILER}\_(static\_){32bit|64bit}}~.\\
}
To clear all such folders, simply run \texttt{cmake\_six clean}.

\subsection{Supported compilers}
Most of the code is written in Fortran, where we require Fortran 2003 support.
The currently supported compilers are:
\begin{itemize}
\item gfortran
\item ifort
\item nagfor
\end{itemize}
Furthermore, some of the support libraries are written in C/C++.
We here support the following compilers:
\begin{itemize}
\item gcc
\item icc / icpc
\item LLVM (OS X)
\end{itemize}
These can be combined, by setting the compiler flags as shown by \texttt{cmake\_six help}.
Note that one may also use whatever compilers are the system default by using the \texttt{defaultcompiler} flag to \texttt{cmake\_six}.

Also note that some of these compilers may only be able to build either 32- or 64-bit executables, either due to limitations in the compiler itself, or due to which libraries have been installed on the machine you are compiling on.

\subsubsection{gfortran and gcc}
The gfortran and gcc compiler is the default on CERN's lxplus system and on most linux systems.
The system version (version 4.4.7) is old, but can be used to build both 32- and 64-bit executables.

However the version installed and loaded by default on lxplus is quite old; it is however possible to load a newer version.
This can be accomplished by running their setup script as\\
{\scriptsize
  \texttt{source /afs/cern.ch/sw/lcg/external/gcc/4.9/\-x86\_64\--slc6\--gcc49\--opt/\-setup.sh}\\
}
where this version is selected to match the Geant4 version that can be loaded as
{\scriptsize
  \texttt{source /afs/cern.ch/sw/lcg/external/geant4/10.3/x86\_64\--slc6\--gcc49\--opt/\-CMake-setup.sh}~.\\
}
Unfortunately, this version only have the libraries for building the 64-bit version of SixTrack.

\todo[inline]{Installing gfortran and libraries for 32- and 64-bit, static/nonstatic builds on Fedora and Ubuntu.}

\subsubsection{ifort and icc/icpc}
Another popular set of compilers are Intel's ifort and icc.
To load these from AFS, simply run their setup script as\\
{\small
  \texttt{source /afs/cern.ch/sw/IntelSoftware/linux/17-all-setup.sh}~.\\
}
These compilers are capable of producing both 32- and 64-bit executables.

However note that if using icc/icpc, it is not possible to compile static or 32-bit executables, since Intel does not provide the required libraries.

If running the Intel compilers on LxPlus, it is reccomended to first load a newer version CMAKE.
This can be accomplished through the commands:\\
{\scriptsize
  \texttt{export PATH=/afs/cern.ch/sw/lcg/contrib/CMake/3.7.0/Linux-x86\_64/bin:\$PATH}\\
  \texttt{export CMAKE\_ROOT=/afs/cern.ch/sw/lcg/contrib/CMake/3.7.0/Linux-x86\_64}
}

\subsubsection{nagfor}

To load nagfor version 6.0 on LxPlus, simply run\\
\texttt{source /afs/cern.ch/sw/fortran/nag/usenag.bash 6.0}\\
before any compilation commands.
Both 32- and 64-bit executables are supported, as well as both static- and dynamic linking.

%In the future, maybe a new-ish Lahey.

\subsection{Common build types}
\label{sec:building:options}
By selecting different sets of flags at compile time, the SixTrack sources can be compiled to different versions, which have different capabilities.
This section lists the most common build types; note that it is often possible to combine these features, e.g. to use collimation together with increased particle numbers.
If the features are incompatible, then this is detected by the cmake script, which will exit with an error message.

\subsubsection{Standard build}
This build type, used for generic tracking and dynamic aperture studies, is selected by using the standard build flags.
In effect, just run \texttt{cmake\_six} without any flags, possibly except for explicitly selecting a compiler.
Note that the standard build uses CRLIBM, described in Section~\ref{sec:building:libs:crlibm}.

\subsubsection{Collimation}
The collimation version of SixTrack is used for collimation studies.
Here, more than 64 particles can be tracked by splitting it up in ``packs''; in total a maximum of \texttt{maxn=}20'000 particles can be used.
Furthermore, the collimators can scatter particles; the scattering angle is determined through a Monte Carlo routine.
The collimation-specific features are controlled by a \texttt{COLL} block in the \texttt{fort.3} input file; please see \todo{ref} for more information.

In order to compile the collimation version, run \texttt{cmake\_six COLLIMAT -CRLIBM}.

\todo[inline]{Sub versions - merlinscatter, hdf5, etc.}

\subsubsection{Checkpoint/restart}
The checkpoint/restart feature in SixTrack allows the simulation to continue from where it stopped after an abort, instead of restarting from the beginning.
This is a vital feature when running on e.g.\ LHC@Home, and works by saving a checkpoint file every \texttt{numlcp} turns.
The feature is included by using the \texttt{CR} build option, i.e.\ \texttt{cmake\_six CR}.

\subsubsection{Increased particle numbers}
It is possible to track more than 64 particles simultaniously in SixTrack; the main limitation for this is the ``binary data files'' fort.90, fort.89, etc., as one such file is written per pair of tracked particles.
In order to work around this, the ``Single Track File'' (STF) functionality was invented, which basically packs the contents of these binary data files into a single file \texttt{singletrackfile.dat}.
This makes it possible to increase the particle number to 2048, and to track this many particles compile with \texttt{cmake\_six STF BIGNPART}.

After this, the limiting factor is the size of the executable's BSS section, which contains the Fortran COMMON blocks, when using the ``small'' code model (2~GB).
In order to break out of this limiation, some large matrices that are only needed for thick tracking can be allocated on demand in heap memory, and this is enabled by the \texttt{DATAMODS} option.
This then makes it possible to track up to 65536 particles when the \texttt{HUGENPART} option is in use; to compile this run \texttt{cmake\_six STF DATAMODS HUGENPART}~.

\todo[inline]{Reference sixtrack meeting slides?}

\subsection{Libraries}
languages, linking, what they do etc.

\subsubsection{CRLIBM}
\label{sec:building:libs:crlibm}

\subsubsection{MerlinScatter}

\subsubsection{HDF5}

\subsubsection{BOINC}

\subsection{Astuce}

\subsection{DAFOR}

\subsection{Building on platforms other than Linux}

\subsubsection{Building on OS X}

\subsubsection{Building on Windows}

msys2

\subsection{Legacy build environments}
make\_six and the makefile

\section{Testing SixTrack}

Note on SSE etc.

\subsection{Running CTEST}

\subsection{Submitting to CDASH}

\subsection{Adding new tests}

\subsection{Legacy test environment ``SixTest''}

\end{document}
