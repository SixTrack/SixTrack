\documentclass[english]{article}
\usepackage[T1]{fontenc}
\usepackage[latin9]{inputenc}
\usepackage{textcomp}
\usepackage{amsmath, amsfonts}
\usepackage{babel}
\usepackage{lscape}
\usepackage{hyperref}

\makeatletter
\@ifpackageloaded{tex4ht}{%
  \def\pgfsysdriver{pgfsys-tex4ht.def}%
}{%
  % only needed inside a class
  \begingroup\expandafter\expandafter\expandafter\endgroup
  \expandafter\ifx\csname HCode\endcsname\relax
  \else
    \def\pgfsysdriver{pgfsys-tex4ht.def}%
  \fi
}
\makeatother

\usepackage{tikz}

\usepackage{cite}
\usepackage{url}


\begin{document}

\author{K. Sjobak, J. Molson}
\title{Compiling, building, and testing SixTrack}
\date{Dec. 2016}

\maketitle

\begin{abstract}
  The SixTrack code supports a wide variety of compile time options,
  compilers, and build environments.
  This document intends to present what is supported, and how to build the most common versions.
  Furthermore, an important sub-task of building SixTrack is to test that the built binary is correct; this is also covered.
\end{abstract}

\tableofcontents
\newpage

\section{Downloading SixTrack}
The last stable release of SixTrack should be found on GitHub, in the SixTrack repository under the SixTrack organization:\\
\url{https://github.com/SixTrack/SixTrack}
In order to download a release

\subsection{Contributing}
forks and stuff.

\section{Building SixTrack}
cmake
\subsection{Supported compilers}
Fortran 2003: gfortran, ifort, nagfor.

In the future, maybe a new-ish Lahey.

\subsection{Build options}


\subsection{Libraries}

\subsection{Astuce}

\subsection{DAFOR}

\subsection{Legacy build environments}
make\_six and the makefile

\section{Testing SixTrack}

\subsection{Running CTEST}

\subsection{Submitting to CDASH}

\subsection{Adding new tests}

\subsection{Legacy test environment}

\end{document}
