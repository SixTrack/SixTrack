\chapter{Versions and Service}

There are two versions: for element by element tracking there is a vector version, and there is a version to produce a one--turn map using the LBL Differential Algebra package.
In both cases the input structure file \# 2 is used to determine if the thick or thin linear element mode has to be used.

To use the power of the Differential Algebra, for instance to calculate the 6D closed orbit in an elegant fashion, the tracking versions may also be equipped with a low order map facility to avoid the otherwise huge demand on memory.

It must be mentioned that in the linear thin lens version dipoles have to be treated in a special way.
See section~\ref{MulBlo} for details.

To convert MAD files into SixTrack input a special conversion program $mad\_6t$~\cite{CONVERTOR} has been developed (see also~\ref{MADT}).

The following subroutines are taken from various packages:

\begin{table}[h]
\caption{External Routines}
\label{T-ExtRou}
\scriptsize \centering
\begin{tabular}{|l|l|l|}
  \hline \rule[-3.75mm]{0mm}{10mm}
  {\bf\large Package} & {\bf\large Routine} & {\bf\large Purpose} \\
  \hline \rule[-3.75mm]{0mm}{10mm} NAGLIB & E04UCF, E04UDM, E04UEF, X04ABF & using internally Normal Forms \\
  \hline \rule[-3.75mm]{0mm}{10mm} HBOOK  & HBOOK2, HDELET, HLIMIT, HTITLE & graphic basics \\
  \hline \rule[-3.75mm]{0mm}{10mm} HPLOT  & HPLAX,  HPLCAP, HPLEND, HPLINT, & graphic options \\
  \cline{2-2} \rule[-3.75mm]{0mm}{10mm}   & HPLOPT, HPLSET, HPLSIZ, HPLSOF &  \\
  \hline \rule[-3.75mm]{0mm}{10mm} HIGZ   & IGMETA, ISELNT, IPM, IPL       & graphic output \\
  \hline
\end{tabular}
\normalsize
\end{table}

All versions can be downloaded from the web.
The project webpage is found at \url{http://sixtrack.web.cern.ch/},
  and primary source repository is located at \url{https://github.com/SixTrack/SixTrack}.
Older versions can be found at \url{http://cern.ch/Frank.Schmidt/Source}.

In case of problems, please see the CERN SixTrack egroups ``sixtrack-users'' and ``sixtrack-developers''.
If these are not accessible to you, you are welcome to contact the coordinators: Riccardo De Maria and Kyrre Sjobak, as well as the original developer Frank Schmidt.
Our contact details are available from the CERN phonebook.

If you think you have found a defect in the program, please create a report on the issue tracker at \url{https://github.com/SixTrack/SixTrack/issues}.
Note that for this to be usefull, you need to describe what the program is doing, what you expected it to do, and an example which demonstrates the unwanted behaviour.
Plase also look through the issues that are already listed, and see if it is known.
If so, you are welcome to add a comment to the issue, which may influence its priority or give additional and useful information to the developers.

The most up to date version of the documentation can always be found on the GitHub repository mentioned above.
Additionally, various older documentation can be found at \url{http://cern.ch/Frank.Schmidt/Documentation/doc.html}.

\subsection{Evolution of SixTrack}

Lastly, we would like to give a short historical overview how the versions of SixTrack have evolved.
\begin{itemize}
  \item {\bf Version 1}
        The first version has been an upgrade of RACETRACK~\cite{RACETRACK} to include the full 6D formalism for long linear elements by G.~Ripken~\cite{Ripken85}.
  \item {\bf Version 2}
        The DA package and the Normal Form techniques~\cite{Berz89,Forest89} have been added to allow the production of high order one turn Taylor maps and their analysis.
        The 6D thin lens formalism~\cite{Ripken95} has also been included to speed up the tracking without appreciable deterioration of the accelerator model for very large Hadron colliders like the LHC.
  \item {\bf Version 3}
        The beam--beam kick \`a la Bassetti and Erskine~\cite{BasErs} has been included together with the 6D part by Hirata et al.~\cite{Hirata}.
        Moreover, this 6D part has been upgraded to include the full 6D linear coupling~\cite{ripbeam}.
        Lastly, the LBL DA package has replaced the original one by Berz, and all operations needed to set up the accelerator structure are now performed with the help of Forest's LieLib package~\cite{DALIE}.
  \item {\bf Version 4}
        \todo{Add stuff}  
  \item {\bf Version 5}
        \todo{Add stuff}  
\end{itemize}

