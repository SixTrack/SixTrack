\begin{titlepage}
\begin{center}\normalsize\scshape
    European Organization for Nuclear Research \\
    CERN BE/ABP
\end{center}
\vspace*{2mm}
\begin{flushright}
    CERN/xx/xx \\
    Updated March 2018
\end{flushright}
\begin{center}\Huge
    \textbf{SixTrack} \\
    \LARGE Version 5.0 \\
    \vspace*{8mm}Single Particle Tracking Code Treating Transverse Motion with Synchrotron Oscillations in a Symplectic Manner \\
    \vspace*{8mm}\textbf{User's Reference Manual}
\end{center}
\begin{center}
    F. Schmidt\\
    \vspace*{4mm}Updated by:
    A.~Alekou,
    V.K.~Berglyd~Olsen,
    R.~De~Maria,
    M.~Fitterer,
    S.~Kostoglou,
    T.~Persson,
    K.~Sjobak,
    J.F.~Wagner,
    and
    S.J.~Wretborn,
\end{center}
\begin{center}\large
    \vspace*{10mm}\textbf{Abstract} \\
\end{center}
The aim of SixTrack is to track two nearby particles taking into account the full six--dimensional phase space including synchrotron oscillations in a symplectic manner.
It allows to predict the long--term dynamic aperture which is defined as the border between regular and chaotic motion.
This border can be found by studying the evolution of the distance in phase space of two initially nearby particles.
Parameters of interest like nonlinear detuning and smear are determined via a post processing of the tracking data.
An analysis of the first order resonances can be done and correction schemes for several of those resonances can be calculated.
Moreover, there is the feature to calculate a one--turn map to very high order and the full six--dimensional case, using LBL differential algebra.
This map allows a subsequent theoretical analysis like normal form procedures which are provided by \'{E}. Forest~\cite{DALIE}.

The linear elements are usually treated as thick elements in SixTrack\@.
In that case there is at least one non--zero length element in the structure file which is not a drift element.
If the accelerator, however, is modelled exclusively with drifts and kicks, SixTrack automatically uses the thin lens formalism according to G.~Ripken \cite{Ripken95}.
A common header of output data and the format of these data has been found for MAD and SixTrack tracking data.

\vfill
\begin{center}
    Geneva, Switzerland \\
    \today
\end{center}

\end{titlepage}
