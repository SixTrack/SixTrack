
\chapter{Organising Tasks} \label{OrgTask}

In this section the input data blocks are described, which are used to
organise the input structure.

\section{Random Fluctuation Starting Number} \label{FluNum}

\subparagraph{Description} If besides mean values for the multipole
errors (Gaussian) random errors should be considered this input data
structure is used to set the start value for the random generator.

\subparagraph{Keyword} FLUC \subparagraph{Number of data lines} 1

\subparagraph{Format} {\em izu0 mmac mout mcut} \/(integers)
\begin{description}
\item [izu0] Start value for the random number generator
\item [mmac] {\em \large -- Sorry: disabled for the time being, i.e.\ 
    mmac is fixed to be 1 --} \/({\small In the vectorised version the
    number of different starting seeds can be varied.  Each seed is
    calculated as \mbox{$ k \times izu0 $} where $k$ runs from 1 to
    {\em mmac} \/which can not exceed 5 to save storage space (see
    list of parameters in Appendix~\ref{DSP}).})
\item {\em mout} A binary switch for various purposes, so all options,
  as described below, can be combined. 
\begin{itemize}
\item {\em mout} \/= 0 : multipole errors internally created
\item {\em mout} \/= 1 : multipole errors read--in from external file
  
  External multipole errors are read--in from file 16 into the array
  of random values. To activate these values one has to set to a value
  of 1 the relevant r.m.s.--positions of the corresponding multipole
  blocks (~\ref{MulCoe}). The systematic components are added as usual
  and multipoles not found in the fort.16 are treated as for ({\em
    mout} \/= 0 ). An error is only detected if there are too few sets
  of multipoles in fort.16.
\item {\em mout} \/= 2: the geometry and strength file is written to
  file \# 4 in the same format as the input file \# 2; the multipole
  coefficients are written to file \# 9; name, misalignments and tilt
  is written to file \# 27 and finally name, random single multipole
  strength and both random transverse misalignments are written to
  file \# 31.
\item {\em mout} \/= 4: Name, horizontal and vertical misalignment and
  also the element tilt are read--in from file \# 8.
\item {\em mout} \/= 8: Name and 3 Random numbers for single kick
  strength and both random transverse misalignments and also the value
  of the tilt are read--in from file \# 30. 
\end{itemize}
\item [mcut] The random distribution can be cut by {\em mcut} \/sigma
  of the distribution. No cuts are applied for {\em mcut = 0}\/.
\end{description}

\subparagraph{Remarks}
\begin{enumerate}
\item The RANECU random generator \cite{RANECU} is used as it produces
  machine independent sequences of random numbers.
\item If the starting point has to be changed or another nonlinear
  element is to be inserted, this can be done without changing the
  once chosen random distribution of errors by using the {\em
    Organisation of Random Numbers} \/input block.
\item The description of an accelerator is fully contained in 4 files:
  fort.2 (geometry), fort.3 (tracking parameters and definition of
  multipole blocks), fort.16 (multipole errors) and fort.30 (random
  numbers of the single multipole kick, the horizontal and vertical
  misalignment and the value of the tilt). This block allows to write
  out the files \# 4, 9, 27, 31 which may serve as the input files \#
  2, 16, 8 and 30 respectively. The file fort.30 superseeds fort.8 if
  both files are read in.
\end{enumerate}

\section{Organisation of Random Numbers} \label{OrgRan}

\subparagraph{Description} Working on a lattice for an accelerator
often requires to introduce new nonlinear elements. In those cases
simply introducing this new element means that the previously chosen
random distribution of the errors will be changed and with it often
the linear parameters. This input data block is mainly used to avoid
this problem by reserving extra random numbers for the new elements.
It also allows to change the observation point without affecting the
machine. The random values of different nonlinear elements including
blocks of multipoles can be set to be equal to allow to vary the
number of nonlinear kicks in one magnet which clearly should have the
same random distribution for each multipolar kick. Finally multipole
sets with different name can be made equal with this input data block.

\subparagraph{Keyword} ORGA \subparagraph{Number of data lines}
variable

\subparagraph{Format} {\em ele1 ele2 ele3} \/The data lines can be set
in three different ways:
\begin{enumerate}
\item Ele1 = ``name'' where name $\ne$ MULT \\
  Ele2 = ignored \\
  Ele3 = ignored \\
  The nonlinear element or multipole set will have its own set of
  random numbers.
\item Ele1 = ``name1'' where name1 $\ne$ MULT \\
  Ele2 = ``name2'' \\
  Ele3 = ignored \\
  The nonlinear element or multipole block Ele1 has the same random
  number set as those of Ele2, if it follows Ele2 as the first
  nonlinear element in the structure list (~\ref{StrInp}).
\item Ele1 = ``MULT'' \\
  Ele2 = ``name2'' \\
  Ele3 = ``name3'' \\
  The multipole set ``name3'' is set to the values of the set
  ``name2''. random errors are not influenced in this case.
\end{enumerate}

\subparagraph{Remarks}
\begin{enumerate}
\item A simple change of the starting point, by placing a ``GO''
  somewhere in structure, used to change the machine optics as the
  random numbers were shifted, too.  Simply calling this block even
  without a data line, will always fix the sequence of random numbers
  to start at the first multipole in the structure.
\item This input data block must follow the definition of the
  multipole block, otherwise multipoles cannot be set equal (option
  3).
\item Do not use the keyword ``MULT'' in the single element list
  (~\ref{SinEle}).
\end{enumerate}

\section{Combination of Elements} \label{ComEle}

\subparagraph{Description} It is often necessary to use several
families of magnetic elements with a certain ratio $ R $ of magnetic
strength to perform corrections like tune adjustment (~\ref{TunVar}),
chromaticity correction (~\ref{ChrCor}) or resonance compensation
(~\ref{ResCom}).  The {\em Combination of Elements} \/input block
allows such a combination of elements.  The maximum number of elements
is defined by the parameter NCOM (see Appendix~\ref{DSP}).

\subparagraph{Keyword} COMB \subparagraph{Number of data lines}
variable

\subparagraph{Format} {\em e0 R1 e1 \dots Rn en}

\begin{description}
\item [e0] Reference element which appears in the input of the
  processing procedure
\item [e1, \dots, en] Elements to be combined with {\em e0}
\item [Rj] Ratio of the magnetic strength of element {\em ej} \/to
  that of element {\em e0}
\end{description}

