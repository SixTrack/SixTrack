
\chapter{General Input} \label{GenInp}

% ================================================================================================================================ %
\section{Main Input Files} \label{InFiles}

SixTrack requires a main input file, which by default must be named \texttt{fort.3}.
Alternatively, if a different file name is desired, it can be given to the SixTrack executable as the first command line argument.
If a geometry file is requested in the main input file (see Section~\ref{ProVer}), it can be given as the second command line argument.
If none is provided, SixTrack will look for a file named \texttt{fort.2}.
These files will be referred to as \texttt{fort.3} and \texttt{fort.2} in the rest of this document.

\bigskip
Note that you can always add the global \texttt{DEBUG} flag to the \texttt{SETTINGS} block to enable echoing back most of the input parameters set in the \texttt{fort.3} file.
The flag is described in Section~\ref{STSett}.
This can not only verify that SixTrack understood and received the values correctly, but it also often echoes back other parameters computed from the input values.
The echoed back lines will start with ``\texttt{INPUT> DEBUG}'', so it is sometimes easier to grep for these lines in the output.
They contain the name of the block where they were parsed, and their line number within the block if available or relevant.

% ================================================================================================================================ %
\section{Program Version} \label{ProVer}

The \textit{Program Version} input block determines if all of the input will be in the input file \texttt{fort.3}, or if the geometry part of the machine (see~\ref{MaGe}) will be in a separate file: \texttt{fort.2}.\index{input block}\index{fort.2}\index{fort.3}
The latter option is useful if tracking parameters are changed, but the geometry part of the input is left as it is.
The geometry part can be produced directly from a MAD-X input file (see~\ref{MADT}).
Note that this line should not have a \texttt{NEXT} keyword after it, and must always be the first line of the file.\index{NEXT}

\bigskip
\begin{tabular}{@{}lp{0.8\linewidth}}
    \textbf{Keyword}    & \texttt{FREE} or \texttt{GEOM} \index{FREE}\index{GEOM}\\
    \textbf{Data lines} & None \\
    \textbf{Format}     & \texttt{keyword comment title} \\
\end{tabular}

\paragraph{Format Description}~

\bigskip
\begin{tabular}{@{}lp{0.8\linewidth}}
    \texttt{keyword} & The first four characters of the first line of the \texttt{fort.3} input file are reserved for the keyword. \texttt{FREE} for free format input with all input in \texttt{fort.3}, and \texttt{GEOM} if the geometry part is in file  \texttt{fort.2}. \\
    \texttt{comment} & Following the first four characters, 8 characters are reserved for comments \\
    \texttt{title}   & The next 60 characters are interpreted as the title printed at the top of the output file \texttt{fort.6}.
\end{tabular}

% ================================================================================================================================ %
\section{Print Selection} \label{PriSel}

The \texttt{PRIN} flag is deprecated, and replaced by the \texttt{PRINT} flag in the \texttt{SETTINGS} block.\index{PRIN}

% ================================================================================================================================ %
\section{Settings} \label{STSett}

The \textit{Print Selection} input block available in earlier version has been replaced with the \texttt{SETTINGS} block.\index{SETT}
This was done to allow for more options related to what output SixTrack produces in \texttt{fort.6}.\index{fort.6}
The \texttt{PRIN} flag is available as one of several options in this block.\index{PRIN}
However, for backwards compatibility, the \texttt{PRIN} flag is still accepted.

\bigskip
\begin{tabular}{@{}lp{0.8\linewidth}}
    \textbf{Keyword}    & \texttt{SETT} \\
    \textbf{Data lines} & Variable
\end{tabular}

\paragraph{Format Description}~

\bigskip
\begin{tabular}{@{}lp{0.8\linewidth}}
    \texttt{PRINT} & This causes the printing of the input data to the output file \texttt{fort.6}. \\
    \texttt{DEBUG} & A global debug flag that causes the majority of the blocks to echo back the value read from the input file after parsing. It may also print out secondary values set based on input values read, or modification made to input values based on other dependencies and criteria.\index{DEBUG}\\
    \texttt{QUIET} & Followed by an integer specifying how ``quiet'' the output should be. A higher value causes less information to be printed back out. If the keyword is not present, the default value is 0, which means it is disabled. If it is present, but the integer value is omitted, it is set to be 1. This flag does not interfere with the \texttt{PRINT} flag.\index{QUIET}\\
    \texttt{PRINT\_DCUM} & This will cause the calculated s-coordinate of each structure element to be printed to the file \texttt{machine\_length.dat}. \\
    \texttt{PARTSUMMARY} & Enable or disable the printing of a particle summary after tracking. The flag takes an optional parameter to explicitly state whether it is ON or OFF. If omitted, it is assumed the user requests it to be ON. If the flag is omitted entirely, the default behaviour is determined by the particle count. If SixTrack is running with 64 particles or less, it is ON by default, otherwise OFF.\index{particle summary}\\
    \texttt{WRITEFORT12} & Enable or disable the writing of \texttt{fort.12} after tracking. The flag takes an optional parameter to explicitly state whether it is ON or OFF. If omitted, it is assumed the user requests it to be ON. If the flag is omitted entirely, the default behaviour is determined by the particle count. If SixTrack is running with 64 particles or less, it is ON by default, otherwise OFF.\index{fort.12}\\
    \texttt{INITIALSTATE} & Followed by either ``binary'' or ``text''. This will write a file before tracking containing the initial state of all particles to either a binary or a text file.\index{initial state} Adding ``ions'' as a second keyword will also dump the additional ion columns (see Section~\ref{hions}). The file header also contains the settings of the reference particle, the 4D and 6D closed orbit, the tunes, and the TA matrix. (Note that the $\mbox{d}p/p_0$ values are not scaled by a factor 1000 in this file.)\index{TA Matrix} \\
    \texttt{FINALSTATE} & Followed by either ``binary'' or ``text''. This will write a file after tracking containing the final state of all particles, including those lost during tracking, to either a binary or a text file.\index{final state} Adding ``ions'' as a second keyword will also dump the additional ion columns (see Section~\ref{hions}). The file header also contains the settings of the reference particle, the 4D and 6D closed orbit, and the TA matrix. (Note that the $\mbox{d}p/p_0$ values are not scaled by a factor 1000 in this file.)\index{TA Matrix} \\
\end{tabular}


% ================================================================================================================================ %
\section{Comment Line} \label{ComLin}

An additional comment can be specified with the \textit{Comment} block.\index{COMM}\index{comment}
The comment will be written to the binary data files (Appendix~\ref{Header}), and will appear in the post processing output as well.

\bigskip
\begin{tabular}{@{}lp{0.8\linewidth}}
    \textbf{Keyword}    & \texttt{COMM}\index{COMM}\index{comment} \\
    \textbf{Data lines} & 1 \\
    \textbf{Format}     & A string of up to 80 characters.
\end{tabular}

% ================================================================================================================================ %
\section{Iteration Errors} \label{IteErr}

For the processing procedures, the number of iterations and the precision to which the processing is to be performed are chosen with the \textit{Iteration Errors} input block.\index{iteration errors}
If the input block is left out, default values will be used.

\bigskip
\begin{tabular}{@{}lp{0.8\linewidth}}
    \textbf{Keyword}    & \texttt{ITER}\index{ITER} \\
    \textbf{Data lines} & 1 to 4 \\
    \textbf{Format}     & Each data line holds three values as in table~\ref{T-IteErr}, except for the fourth line where the horizontal and vertical aperture limits can be additionally specified. This has been added to avoid artificial crashes for special machines.
\end{tabular}

\begin{table}[h]
    \caption{Iteration Errors\index{chromaticity}\index{dispersion}\index{aperture limit}\index{closed orbit}}
    \label{T-IteErr}
    \centering
    \renewcommand{\arraystretch}{1.5}
    \begin{tabular}{|l|l|l|>{\raggedright\arraybackslash}p{10.2cm}|}
        \hline
        \rowcolor{blue!30}
        \textbf{Variable} & \textbf{Type} & \textbf{Default} & \textbf{Description} \\
        \rowcolor{gray!15}
        \multicolumn{4}{|l|}{Data Line 1} \\
        \hline
        \texttt{ITCO} & \texttt{int} & \texttt{50}    & Number of Iterations for closed orbit calculation. \\
        \hline
        \texttt{DMA}  & \texttt{dbl} & \texttt{1e-12} & Demanded Precision of closed orbit displacements. \\
        \hline
        \texttt{DMAP} & \texttt{dbl} & \texttt{1e-15} & Demanded Precision of derivative of closed orbit displacements. \\
        \hline
        \rowcolor{gray!15}
        \multicolumn{4}{|l|}{Data Line 2} \\
        \hline
        \texttt{ITQV} & \texttt{int} & \texttt{10}    & Number of Iterations for Q Adjustment. \\
        \hline
        \texttt{DKQ}  & \texttt{dbl} & \texttt{1e-10} & Variations of quadrupole strengths. \\
        \hline
        \texttt{DQQ}  & \texttt{dbl} & \texttt{1e-10} & Demanded Precision of tunes. \\
        \hline
        \rowcolor{gray!15}
        \multicolumn{4}{|l|}{Data Line 3} \\
        \hline
        \texttt{ITCRO} & \texttt{int} & \texttt{10}    & Number of Iterations for chromaticity correction. \\
        \hline
        \texttt{DSM0}  & \texttt{dbl} & \texttt{1e-10} & Variations of sextupole strengths. \\
        \hline
        \texttt{DECH}  & \texttt{dbl} & \texttt{1e-10} & Demanded Precision of chromaticity correction. \\
        \hline
        \rowcolor{gray!15}
        \multicolumn{4}{|l|}{Data Line 4} \\
        \hline
        \texttt{DE0} & \texttt{dbl} & \texttt{1e-9} & Variations of momentum spread for chromaticity calculation. \\
        \hline
        \texttt{DED} & \texttt{dbl} & \texttt{1e-9} & Variations of momentum spread for evaluation of dispersion. \\
        \hline
        \texttt{DSI} & \texttt{dbl} & \texttt{1e-9} & Demanded Precision of desired orbit r.m.s. value; compensation of resonance width. \\
        \hline
        \texttt{APER(1)} & \texttt{dbl} & \texttt{1000 [mm]} & Demanded Precision of horizontal aperture limit. \\
        \hline
        \texttt{APER(2)} & \texttt{dbl} & \texttt{1000 [mm]} & Demanded Precision of vertical aperture limit. \\
        \hline
    \end{tabular}
\end{table}

% ================================================================================================================================ %
\section{MAD-X to SixTrack Conversion} \label{MADT}

A converter has been developed~\cite{CONVERTOR}, which is directly linked to MAD-X\@.\index{MAD-X}
It produces the geometry file \texttt{fort.2}; an appendix to the parameter file \texttt{fort.3}, which defines which of the multipole errors are switched on; the error file \texttt{fort.16}, and the file \texttt{fort.8} which holds the transverse misalignments and the tilt of the non-linear kick elements.\index{fort.8}\index{fort.16}\index{multipole errors}
It also produce a file \texttt{fort.34} with linear lattice functions, phase advances and multipole strengths needed for resonance calculations for the program \textit{SODD}~\cite{SODD}.\index{fort.34}\index{resonance calculations}

In addition, the flag \texttt{aperture} will produce an aperture limitations file.
The flag \texttt{multicol} will produce an alternative \texttt{fort.2} file with more information on the machine structure (see Section~\ref{stru:multicol}).
