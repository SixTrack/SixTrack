\chapter{Data Structure of the Data Files} \label{Header}

A common data structure for the programs MAD-X and SixTrack is agreed on\footnote{
Historically, SixTrack wrote one such file per particle pair (files \texttt{fort.90}, \texttt{fort.89}, \ldots), however recent versions of SixTrack always produce a single \texttt{singletrackfile.dat} containing all the data, by interleaving the contents of the old files in one large file as discussed below.
Files of the old (MAD-X and Sussix compatible) format can be generated from this new format by running the tool \texttt{splitSingletrack} in the folder containing the \texttt{singletrackfile.dat}.
This splits the combined \texttt{singletrackfile.dat} into one file for each pair.

This chapter describes the old format used for single particle pairs, followed by a description of the built-in postprocessing tool output.}.
Besides some minor differences this allows a straightforward post-processing of data from either program.
Each binary data file has a header which holds a description of the run with comments, tracking parameters and 50 additional parameters for future purposes, six of which are already specified in SixTrack.

\begin{table}[h]
    \caption{Header of the Binary Data Files}
    \label{T-HDF}
    \centering
    \begin{tabular}{|c|c|>{\raggedright\arraybackslash}p{8cm}|}
        \hline
        \rowcolor{blue!30}
        \textbf{Data Type} & \textbf{Bytes} & \textbf{Description} \\
        \hline
        Character & 80 & General title of the run \\
        \hline
        Character & 80 & Additional title \\
        \hline
        Character & 8 & Date \\
        \hline
        Character & 8 & Time \\
        \hline
        Character & 8 & Program name \\
        \hline
        Integer & 4 & First particle in the file \\
        \hline
        Integer & 4 & Last particle in the file \\
        \hline
        Integer & 4 & Total number of particles \\
        \hline
        Integer & 4 & Code for dimensionality of phase space 1,2,4 are hor., vert.\ and longitudinal respectively \\
        \hline
        Integer & 4 & Projected number of turns \\
        \hline
        Float & 8 & Horizontal Tune \\
        \hline
        Float & 8 & Vertical Tune \\
        \hline
        Float & 8 & Longitudinal Tune \\
        \hline
        Float & 6 * 8 & Closed Orbit vector \\
        \hline
        Float & 6 * 8 & Dispersion vector \\
        \hline
        Float & 36 * 8 & Six-dimensional transfer map \\
        \hline
        \rowcolor{gray!15}
        \multicolumn{3}{|l|}{50 additional parameters} \\
        \hline
        Float & 8 & Maximum number of different seeds \\
        \hline
        Float & 8 & Actual seed number \\
        \hline
        Float & 8 & Starting value of the seed \\
        \hline
        Float & 8 & Number of turns in the reverse direction \\
        & & (IBM only) \\
        \hline
        Float & 8 & Correction factor for the Lyapunov ($\sigma = s - v_0 \times t$) \\
        \hline
        Float & 8 & Correction factor for the Lyapunov (\mbox{$\Delta p/p_0$}) \\
        \hline
        Float & 8 & Start turn number for ripple prolongation \\
        \hline
        Float & 43 * 8 & Dummies \\
        \hline
    \end{tabular}
\end{table}

\clearpage

Following this header the tracking data are written in n samples of mine numbers preceded by the turn number.
In the MAD-X format, the number of samples in is not restricted, whilst SixTrack writes only up to two samples for the two particles for the Lyapunov exponent method.
Up to 64 particles (two per file) can be treated in the vectorised version of SixTrack.

\begin{table}[h]
    \caption{Format of the Binary Data}
    \label{T-FBD}
    \centering
    \begin{tabular}{|c|c|>{\raggedright\arraybackslash}p{8cm}|}
        \hline
        \rowcolor{blue!30}
        \textbf{Data Type} & \textbf{Bytes} & \textbf{Description} \\
        \hline
        Integer & 4 & Turn number \\
        \hline
        \rowcolor{gray!15}
        \multicolumn{3}{|l|}{One or two samples of 9 values are following} \\
        \hline
        Integer & 4 & Particle number \\
        \hline
        Float & 8 & Angular distance in phase space ( $ <= 1 $ ) \\
        \hline
        Float & 8 & x (mm) \\
        \hline
        Float & 8 & $x'$ (mrad)\\
        \hline
        Float & 8 & y (mm) \\
        \hline
        Float & 8 & $y'$ (mrad) \\
        \hline
        Float & 8 & Path-length ($\sigma = s - v_0 \times t$) (mm) \\
        \hline
        Float & 8 & Relative momentum deviation \mbox{$ \Delta p/p_0$}\\
        \hline
        Float & 8 & Energy (MeV) \\
        \hline
    \end{tabular}
\end{table}

Note that in case the ``Single Track File'' option is enabled at compile time, multiple of these files (normally one per particle pair) are interleaved in a single file.
This is done by writing first all headers in order (i.e. first the header for initial particle/final particle 1/2, then 3/4, 5/6 etc.) and then the same for the tracking data.
The ``total number of particles'' field can always be read from the first header record, which gives the number of header records present in the file.
The two file formats are equivalent, i.e. they contain exactly the same data, and it is thus possible to convert losslessly between them.

Some of the post processing data is written in Ascii format to file \texttt{fort.10}.
This can be used for instance for plotting purposes.
Each time the post processing routine is called 60 double precision numbers (some of them still dummy) are added to the file.

The file with the errors (in: \texttt{fort.16}, out: \texttt{fort.9}) has the following format:

\bigskip
\begin{tabular}{ll}
    \textbf{first line} & name of element; \\
    \textbf{line 2--7}  & normal multipoles order 1--18; \\
    \textbf{line 8}     & normal multipoles of order 19 and 20; \\
    \textbf{line 9--14} & skew multipoles order 1--18; \\
    \textbf{line 15}    & skew multipoles of order 19 and 20.
\end{tabular}

\bigskip
The strength definition is according to block~\ref{MulCoe} and to be effective in \texttt{fort.3}.
The random values of the corresponding multipole block have to be set to 1.0.
A word of caution: when writing on file \texttt{fort.9} the \textit{total} multipole strength is used, i.e.\ systematic and random part combined.
File \texttt{fort.16} and \texttt{fort.9} might therefore be different.
When using \texttt{fort.9} as input (\texttt{fort.16}), the systematic part in \texttt{fort.3} has to be set to 0.0.

Misalignment and tilt are in file \texttt{fort.8} and \texttt{fort.27} as input and output respectively.
The format is (\texttt{a16,2x,1p,2d14.6,d17.9}), i.e.\ name, horizontal misalignment, vertical misalignment and tilt.
The misalignment is in units of [mm] the tilt in units of [mrad].
The files \texttt{fort.30} (in) and \texttt{fort.31} (out) have the random single non-linear element kick, misalignments and tilt in the format: (\texttt{a8,1p,d19.11,2d14.6,d17.9}).
Misalignment and tilt in file \texttt{fort.8} or \texttt{fort.30} is automatically activated, while the random strength (strength definition same as in block~\ref{SinEle}) needs an entry in the fourth column in the geometry file \texttt{fort.2}.
Files \texttt{fort.28} and \texttt{fort.29} hold integer counter and closed orbit displacement at a horizontal or vertical monitor respectively.

\newcounter{dst} \setcounter{dst}{0}

\begin{center}
\begin{longtable}{|c|>{\raggedright\arraybackslash}p{12cm}|}
    \caption{Post Processing Data} \label{T-PPD}\\
    \hline

    \rowcolor{blue!30}
    \textbf{Column} & \textbf{Description} \\
    \hline
    \endfirsthead

    \hline
    \rowcolor{blue!30}
    \textbf{Column} & \textbf{Description} \\
    \hline
    \endhead

    \hline \stepcounter{dst}
    \thedst & Maximum turn number \\
    \hline \stepcounter{dst}
    \thedst & Stability Flag (0=stable, 1=lost) \\
    \hline \stepcounter{dst}
    \thedst & Horizontal Tune \\
    \hline \stepcounter{dst}
    \thedst & Vertical Tune \\
    \hline \stepcounter{dst}
    \thedst & Horizontal $\beta$-function \\
    \hline \stepcounter{dst}
    \thedst & Vertical $\beta$-function \\
    \hline \stepcounter{dst}
    \thedst & Horizontal amplitude $1^{st}$ particle\\
    \hline \stepcounter{dst}
    \thedst & Vertical amplitude $1^{st}$ particle\\
    \hline \stepcounter{dst} \thedst & Relative momentum deviation \mbox{$\Delta p/p_0$}\\
    \hline \stepcounter{dst}
    \thedst & Final distance in phase space \\
    \hline \stepcounter{dst}
    \thedst & Maximum slope of distance in phase space \\
    \hline \stepcounter{dst}
    \thedst & Horizontal detuning \\
    \hline \stepcounter{dst}
    \thedst & Spread of horizontal detuning \\
    \hline \stepcounter{dst}
    \thedst & Vertical detuning \\
    \hline \stepcounter{dst}
    \thedst & Spread of vertical detuning \\
    \hline \stepcounter{dst}
    \thedst & Horizontal factor to nearest resonance \\
    \hline \stepcounter{dst}
    \thedst & Vertical factor to nearest resonance \\
    \hline \stepcounter{dst}
    \thedst & Order of nearest resonance \\
    \hline \stepcounter{dst}
    \thedst & Horizontal smear \\
    \hline \stepcounter{dst}
    \thedst & Vertical smear \\
    \hline \stepcounter{dst}
    \thedst & Transverse smear \\
    \hline \stepcounter{dst}
    \thedst & Survived turns $1^{st}$ particle \\
    \hline \stepcounter{dst}
    \thedst & Survived turns $2^{nd}$ particle \\
    \hline \stepcounter{dst}
    \thedst & Starting seed for random generator \\
    \hline \stepcounter{dst}
    \thedst & Synchrotron tune \\
    \hline \stepcounter{dst}
    \thedst & Horizontal amplitude $2^{nd}$ particle\\
    \hline \stepcounter{dst}
    \thedst & Vertical amplitude $2^{nd}$ particle\\
    \hline \stepcounter{dst}
    \thedst & Minimum horizontal amplitude\\
    \hline \stepcounter{dst}
    \thedst & Mean horizontal amplitude\\
    \hline \stepcounter{dst}
    \thedst & Maximum horizontal amplitude\\
    \hline \stepcounter{dst}
    \thedst & Minimum vertical amplitude\\
    \hline \stepcounter{dst}
    \thedst & Mean vertical amplitude\\
    \hline \stepcounter{dst}
    \thedst & Maximum vertical amplitude\\
    \hline \stepcounter{dst}
    \thedst & Minimum horizontal amplitude (linear decoupled)\\
    \hline \stepcounter{dst}
    \thedst & Mean horizontal amplitude (linear decoupled)\\
    \hline \stepcounter{dst}
    \thedst & Maximum horizontal amplitude (linear decoupled)\\
    \hline \stepcounter{dst}
    \thedst & Minimum vertical amplitude (linear decoupled)\\
    \hline \stepcounter{dst}
    \thedst & Mean vertical amplitude (linear decoupled)\\
    \hline \stepcounter{dst}
    \thedst & Maximum vertical amplitude (linear decoupled)\\
    \hline \stepcounter{dst}
    \thedst & Minimum horizontal amplitude (nonlinear decoupled)\\
    \hline \stepcounter{dst}
    \thedst & Mean horizontal amplitude (nonlinear decoupled)\\
    \hline \stepcounter{dst}
    \thedst & Maximum horizontal amplitude (nonlinear decoupled)\\
    \hline \stepcounter{dst}
    \thedst & Minimum vertical amplitude (nonlinear decoupled)\\
    \hline \stepcounter{dst}
    \thedst & Mean vertical amplitude (nonlinear decoupled)\\
    \hline \stepcounter{dst}
    \thedst & Maximum vertical amplitude (nonlinear decoupled)\\
    \hline \stepcounter{dst}
    \thedst & Emittance Mode I\\
    \hline \stepcounter{dst}
    \thedst & Emittance Mode II\\
    \hline \stepcounter{dst}
    \thedst & Secondary horizontal $\beta$-function\\
    \hline \stepcounter{dst}
    \thedst & Secondary vertical $\beta$-function\\
    \hline \stepcounter{dst}
    \thedst & $Q'_x$\\
    \hline \stepcounter{dst}
    \thedst & $Q'_y$\\
    \hline \stepcounter{dst}
    \thedst--58 & Dummy \\
    \hline
    59--60 & Internal use\\
    \hline
\end{longtable}
\end{center}

As an option the 4D linear parameters can be dumped to file \texttt{fort.11} when the linear optics block~\ref{LinOpt} is activated.
This can be used for instance for a post-processing of linear coupling.
25 values are written in a binary format.

\newcounter{dlo} \setcounter{dlo}{0}

\bigskip
\begin{center}
\begin{longtable}{|c|>{\raggedright\arraybackslash}p{12cm}|}
    \caption{4D Linear Parameters} \label{t-4lp}\\
    \hline

    \rowcolor{blue!30}
    \textbf{Column} & \textbf{Description} \\
    \hline
    \endfirsthead

    \hline
    \rowcolor{blue!30}
    \textbf{Column} & \textbf{Description} \\
    \hline
    \endhead

    \hline \stepcounter{dlo}
    \thedlo & Name of the element \\
    \hline \stepcounter{dlo}
    \thedlo & Longitudinal Position [m] \\
    \hline \stepcounter{dlo}
    \thedlo & Horizontal phase advance \\
    \hline \stepcounter{dlo}
    \thedlo & Vertical phase advance \\
    \hline \stepcounter{dlo}
    \thedlo & Primary horizontal $\beta$-function [m] \\
    \hline \stepcounter{dlo}
    \thedlo & Secondary horizontal $\beta$-function [m] \\
    \hline \stepcounter{dlo}
    \thedlo & Secondary vertical $\beta$-function [m] \\
    \hline \stepcounter{dlo}
    \thedlo & Primary vertical $\beta$-function [m] \\
    \hline \stepcounter{dlo}
    \thedlo & Primary horizontal $\alpha$-function [rad] \\
    \hline \stepcounter{dlo}
    \thedlo & Secondary horizontal $\alpha$-function [rad] \\
    \hline \stepcounter{dlo}
    \thedlo & Secondary vertical $\alpha$-function [rad] \\
    \hline \stepcounter{dlo}
    \thedlo & Primary vertical $\alpha$-function [rad] \\
    \hline \stepcounter{dlo}
    \thedlo & Primary horizontal $\gamma$-function [m] \\
    \hline \stepcounter{dlo}
    \thedlo & Secondary horizontal $\gamma$-function [m] \\
    \hline \stepcounter{dlo}
    \thedlo & Secondary vertical $\gamma$-function [m] \\
    \hline \stepcounter{dlo}
    \thedlo & Primary vertical $\gamma$-function [m]\\
    \hline \stepcounter{dlo}
    \thedlo & Primary horizontal phase of x-coordinate [pi] \\
    \hline \stepcounter{dlo}
    \thedlo & Secondary horizontal phase of x-coordinate [pi] \\
    \hline \stepcounter{dlo}
    \thedlo & Secondary vertical phase of y-coordinate [pi] \\
    \hline \stepcounter{dlo}
    \thedlo & Primary vertical phase of y-coordinate [pi] \\
    \hline \stepcounter{dlo}
    \thedlo & Primary horizontal phase of $x^\prime$-coordinate [pi] \\
    \hline \stepcounter{dlo}
    \thedlo & Secondary horizontal phase of $x^\prime$-coordinate [pi] \\
    \hline \stepcounter{dlo}
    \thedlo & Secondary vertical phase of $y^\prime$-coordinate [pi] \\
    \hline \stepcounter{dlo}
    \thedlo & Primary vertical phase of $y^\prime$-coordinate [pi] \\
    \hline \stepcounter{dlo}
    \thedlo & Coupling angle [pi] \\
    \hline \stepcounter{dlo}
    \thedlo & $D_x$ [mm]\\
    \hline \stepcounter{dlo}
    \thedlo & $D^\prime_x$ [mrad]\\
    \hline \stepcounter{dlo}
    \thedlo & $D_y$ [mm]\\
    \hline \stepcounter{dlo}
    \thedlo & $D^\prime_y$ [mrad]\\
    \hline
\end{longtable}
\end{center}

When external multipole errors are read in (see section~\ref{FluNum}), the program expects a complete list of magnet errors to file \texttt{fort.16}.
The format of each set of multipole errors is given in table~\ref{T-XME}.
The definition of the multipole coefficients should be as described in section~\ref{MulCoe}.

\newcounter{dsu} \setcounter{dsu}{0}

\bigskip
\begin{center}
\begin{longtable}{|c|>{\raggedright\arraybackslash}p{12cm}|}
    \caption{Format of file with external errors, \texttt{fort.16}, and internal errors written to \texttt{fort.9}} \label{T-XME}\\
    \hline

    \rowcolor{blue!30}
    \textbf{Row} & \textbf{Description} \\
    \hline
    \endfirsthead

    \hline
    \rowcolor{blue!30}
    \textbf{Row} & \textbf{Description} \\
    \hline
    \endhead

    \hline \stepcounter{dsu}
    \thedsu & Name of multipole set \\
    \hline \stepcounter{dsu}
    \thedsu & $B_1 \ B_2 \ B_3$\\
    \hline \stepcounter{dsu}
    \thedsu & $B_4 \ B_5 \ B_6$\\
    \hline \stepcounter{dsu}
    \thedsu & $B_7 \ B_8 \ B_9$\\
    \hline \stepcounter{dsu}
    \thedsu & $B_{10} \ B_{11} \ B_{12}$\\
    \hline \stepcounter{dsu}
    \thedsu & $B_{13} \ B_{14} \ B_{15}$\\
    \hline \stepcounter{dsu}
    \thedsu & $B_{16} \ B_{17} \ B_{18}$\\
    \hline \stepcounter{dsu}
    \thedsu & $B_{19} \ B_{20} $\\
    \hline \stepcounter{dsu}
    \thedsu & $A_1 \ A_2 \ A_3$\\
    \hline \stepcounter{dsu}
    \thedsu & $A_4 \ A_5 \ A_6$\\
    \hline \stepcounter{dsu}
    \thedsu & $A_7 \ A_8 \ A_9$\\
    \hline \stepcounter{dsu}
    \thedsu & $A_{10} \ A_{11} \ A_{12}$\\
    \hline \stepcounter{dsu}
    \thedsu & $A_{13} \ A_{14} \ A_{15}$\\
    \hline \stepcounter{dsu}
    \thedsu & $A_{16} \ A_{17} \ A_{18}$\\
    \hline \stepcounter{dsu}
    \thedsu & $A_{19} \ A_{20}$\\
    \hline
\end{longtable}
\end{center}

With the parameter \texttt{mout} set to 2 or 3 in the ``Random Fluctuation'' block (\ref{FluNum}), the internally used multipoles are written to file \texttt{fort.9} in the same format as above.
This file can therefore be used as an input \texttt{fort.16} file for a subsequent run.

The file \texttt{fort.34} is written when the ``Linear Optic Block'' (see section~\ref{LinOpt}) is invoked with the \texttt{ELEMENT 0} option.

\setcounter{dsu}{0}

\bigskip
\begin{center}
\begin{longtable}{|c|>{\raggedright\arraybackslash}p{12cm}|}
    \caption{Format of file \texttt{fort.34} for detuning and distortion calculation with external program ``SODD''~\cite{SODD}} \label{T-SODD}\\
    \hline

    \rowcolor{blue!30}
    \textbf{Column} & \textbf{Description} \\
    \hline
    \endfirsthead

    \hline
    \rowcolor{blue!30}
    \textbf{Column} & \textbf{Description} \\
    \hline
    \endhead

    \hline \stepcounter{dsu}
    \thedsu & Longitudinal position [m] \\
    \hline \stepcounter{dsu}
    \thedsu & Type \texttt{n} of Multipole ($n >$ 0 $=>$ erect, $n <$ 0 $=>$ skew) \\
    \hline \stepcounter{dsu}
    \thedsu & Multipole strength [$\mathrm{mrad} \cdot \mathrm{mm}^{(1-|n|)}$] \\
    \hline \stepcounter{dsu}
    \thedsu & Horizontal $\beta$-function [m] \\
    \hline \stepcounter{dsu}
    \thedsu & Vertical $\beta$-function [m] \\
    \hline \stepcounter{dsu}
    \thedsu & Horizontal phase advance \\
    \hline \stepcounter{dsu}
    \thedsu & Vertical phase advance \\
    \hline
\end{longtable}
\end{center}

The last line serves as the end of the structure:
Length of the accelerator, fake name \texttt{END}, fake type \texttt{100}, $\beta$ functions and phase advances at the end of the accelerator for the horizontal and vertical plane respectively.
