
\chapter{Initial Conditions for Tracking} \label{InitCondTrack}

For the study of non-linear systems, the choice of initial conditions is of crucial importance.\index{initial conditions}
The input structure for the initial conditions was therefore organised in such a way as to allow for maximum flexibility.
SixTrack is optimised to reach the largest possible number of turns.
In order to derive the Lyapunov exponent\index{Lyapunov exponent}, and thereby to distinguish between regular and chaotic motion, the particle has a close by companion particle.
Moreover, experience has shown that varying only the amplitude while keeping the phases constant is sufficient to understand the non-linear dynamics, as a subsequent detailed post-processing allows to find the dependence of the parameter of interest on these phases.

A number of features have over time been deprecated or replaced by other modules in SixTrack.
Therefore there are a number of parameters that are no longer in use, but nevertheless have to be set to dummy values.
The Simulation (\texttt{SIMU}) and Distribution (\texttt{DIST}) blocks are intended to replace all of the blocks in this chapter.
These blocks take keyword/value sets instead of blocks of numbers, and are therefore easier to maintain.
The \texttt{SIMU} block (Section~\ref{Input:SIMU}) can currently be used to replace the \texttt{TRAC}, \texttt{INIT}, and \texttt{HION} blocks (Section~\ref{TraPar}), but note that the implementation may still have bugs, and is therefore considered experimental\index{SIMU}.

% ================================================================================================================================ %
\section{Simulation Parameters} \label{Input:SIMU}

\textcolor{notered}{\textbf{Note:} This input block is experimental. It provides an alternative interface to the most used parameters of the \texttt{TRAC}, \texttt{INIT}, and \texttt{HION} blocks, and is intended to be used in combination with the \texttt{DIST} block.}

\bigskip
The Simulation block (\texttt{SIMU}) is intended to take the main simulation parameters, and replaces the \texttt{TRAC}, \texttt{INTI}, and \texttt{HION} blocks\index{SIMU}\index{Simualtion Block}.
If the \texttt{SIMU} block is present in \texttt{fort.3}, these blocks cannot be present.

If the reference particle mass is set in the \texttt{SIMU} block, the value provided in the \texttt{SYNC} block is ignored.

\bigskip
\begin{tabular}{@{}lp{0.7\linewidth}}
    \textbf{Keyword}    & \texttt{SIMU}\index{SIMU} \\
    \textbf{Data lines} & Variable \\
    \textbf{Format}     & Keyword/value format. See Table~\ref{Table:SIMU}.
\end{tabular}

\bigskip
Some settings provided by the \texttt{TRAC} and \texttt{INIT} blocks are not supported by the \texttt{SIMU} block.
These are listed below.
The option to add closed orbit to generated particles is only supported for the amplitude scan, which is not supported by the \texttt{SIMU} block.
If you need this feature, please use the old input format.

\bigskip
\begin{tabular}{@{}llp{0.7\linewidth}}
    \texttt{TRAC} & \texttt{ntwin}          & Fixed to a value of 2.\\
    \texttt{TRAC} & \texttt{idy(1), idy(2)} & Fixed to a value of 1.\\
    \texttt{TRAC} & \texttt{idfor}          & Fixed to a value of 1.\\
    \texttt{TRAC} & \texttt{amp(1), amp0}   & Fixed to a value of 0.\\
\end{tabular}

\begin{center}
\setlength\LTleft{0pt}
\setlength\LTright{0pt}
\begin{longtable}{@{\extracolsep{\fill}}|l|p{10cm}|l|}
    \caption{Available arguments in the SIMU block.}
    \label{Table:SIMU} \\*
    \hline
    \rowcolor{blue!30}
    \textbf{Keyword} & \textbf{Argument(s)} & \textbf{Default} \\*
    \hline
    \endfirsthead

    \hline
    \rowcolor{blue!30}
    \textbf{Keyword} & \textbf{Argument(s)} & \textbf{Default} \\*
    \endhead

    \rowcolor{gray!15}
    \multicolumn{3}{|c|}{(The table continues on the next page)}\\*
    \hline
    \endfoot

    \hline
    \endlastfoot

    \rowcolor{blue!15}
    \multicolumn{3}{|c|}{\textbf{Particles and Turns}}\\*
    \hline

    \rowcolor{gray!15}
    \texttt{PARTICLES} & \texttt{n\_particles(int)} & \texttt{0}\\*
    \hline
    \multicolumn{3}{|>{\raggedright}p{\textwidth}|}{%
        The number of particles to be tracked.
        The value must be an even number.
        This is due to several parts of SixTrack dealing with particles as pairs.
        \index{particles}
    } \\*
    \hline

    \rowcolor{gray!15}
    \texttt{TURNS} & \texttt{forward(int) [backward(int)]} & \texttt{0 0} \\*
    \hline
    \multicolumn{3}{|>{\raggedright}p{\textwidth}|}{%
        The number of turns in the forward, and optionally, backward direction.
        \index{turns}
    } \\*
    \hline

    \rowcolor{gray!15}
    \texttt{CRPOINT} & \texttt{interval(int)} & \texttt{1000} \\*
    \hline
    \multicolumn{3}{|>{\raggedright}p{\textwidth}|}{%
        How often to write checkpoint files.
        This parameter is ignored if SixTrack was not built with checkpoint/restart functionality.
        Checkpoint files are always written on turn 1, then with the interval specified here, and then a last time at the end of tracking.
        \index{checkpoint/restart}\index{CR}
    } \\*

    \hline
    \rowcolor{blue!15}
    \multicolumn{3}{|c|}{\textbf{Reference Particle}}\\*
    \hline

    \rowcolor{gray!15}
    \texttt{REF\_ENERGY} & \texttt{energy(float,MeV)} & \texttt{0.0}\\*
    \hline
    \multicolumn{3}{|>{\raggedright}p{\textwidth}|}{%
        The reference particle energy in MeV.
        \index{reference particle}\index{reference energy}
    } \\*
    \hline

    \rowcolor{gray!15}
    \texttt{REF\_PARTICLE} & \texttt{mass(float,MeV) [charge A Z]} & \texttt{938.271998 1 1 1}\\*
    \hline
    \rowcolor{gray!15}
    \texttt{REF\_PARTICLE} & \texttt{name [charge A Z]} & \texttt{proton 1 1 1}\\*
    \hline
    \multicolumn{3}{|>{\raggedright}p{\textwidth}|}{%
        The reference mass can either be provided as a value in MeV, or as a named particle.
        Currently this can only be set to ``proton''.
        This value defaults to the proton mass set in the SixTrack physical constants module in \texttt{source/constants.f90}.
        Optionally, the reference particle charge and atomic mass (A), and atomic number (Z) can be set.
        If A is set, Z must also be set.
        If only charge is set, Z is set to the same value.
        The default values are all 1 (proton).
        \index{reference particle}\index{reference mass}
    } \\*
    \hline

    \rowcolor{gray!15}
    \texttt{PDG\_YEAR} & \texttt{year(int)} & \texttt{2002}\\*
    \hline
    \multicolumn{3}{|>{\raggedright}p{\textwidth}|}{%
        This can be used to set the PDG year to use for the mass if a name is provided in \texttt{REF\_PARTICLE}.
        The default value is the 2002 proton mass, the other value currently supported is 2018.
        More will be added in the future.
        Note that this value affects how the proton radius is calculated as it uses the PDG year to select the relevant constant for calculating this.
        Even when setting a reference particle mass in MeV, the PDG year is used for this.
        \index{reference particle}\index{PDG}\index{proton radius}
    } \\*
    \hline

    \rowcolor{blue!15}
    \multicolumn{3}{|c|}{\textbf{Lattice and Optics}}\\*
    \hline

    \rowcolor{gray!15}
    \texttt{LATTICE} & \texttt{thin|thick 4D|6D} & \texttt{thin 4D} \\*
    \hline
    \multicolumn{3}{|>{\raggedright}p{\textwidth}|}{%
        The first argument must be either \texttt{thick} or \texttt{thin}, and this must match the content of the geometry file.
        The second argument must be either \texttt{4D} or \texttt{6D}.
        These arguments are not case sensitive.
        When 6D tracking is requested, closed orbit and optical functions at the starting point are calculated using the differential algebra package.
        \index{thick tracking}\index{thin tracking}\index{4D}\index{6D}
    } \\*
    \hline

    \rowcolor{gray!15}
    \texttt{OPTICS} & \texttt{first(int) last(int)} & \texttt{1 nblz} \\*
    \hline
    \multicolumn{3}{|>{\raggedright}p{\textwidth}|}{%
        Start and stop structure element index for optics calculation.
        If set to 0 or omitted, the optics calculation defaults to the full machine.
        \index{optics calculation}
    } \\*
    \hline

    \rowcolor{blue!15}
    \multicolumn{3}{|c|}{\textbf{Closed Orbit}}\\*
    \hline

    \rowcolor{gray!15}
    \texttt{6D\_CLORB} & \texttt{on|off} & \texttt{off} \\*
    \hline
    \multicolumn{3}{|>{\raggedright}p{\textwidth}|}{%
        Compute the 6D closed orbit.
        If the simulation is running 4D, this option is ignored.
        \index{closed orbit}
    } \\*
    \hline

    \rowcolor{gray!15}
    \texttt{INIT\_CLORB} & \texttt{on|off} & \texttt{off} \\*
    \hline
    \rowcolor{gray!15}
    \texttt{INIT\_CLORB} & \texttt{x xp y yp [sigma dpsv] (float)} & \texttt{6 * 0.0} \\*
    \hline
    \multicolumn{3}{|>{\raggedright}p{\textwidth}|}{%
        Initialise closed orbit.
        This keyword can be called either with a flag, in which case it turns on or off the reading of a closed orbit suggestion from file \texttt{fort.33}, or it can provide 4 or 6 values for the closed orbit suggestion.
        If omitted, the values are initialised to zero, and the 4D closed orbit is used to seed the first four values of the 6D closed orbit.
        These settings are ignored when running in 4D.
        \index{closed orbit}\index{6D}
    } \\*
    \hline

    \rowcolor{blue!15}
    \multicolumn{3}{|c|}{\textbf{Particle and Track Files}}\\*
    \hline

    \rowcolor{gray!15}
    \texttt{READ\_FORT13} & \texttt{on|off} & \texttt{off} \\*
    \hline
    \multicolumn{3}{|>{\raggedright}p{\textwidth}|}{%
        Read the particle distribution from file \texttt{fort.13}.
        This file is not intended for reading an initial distribution, but for continuing tracking from a previous simulation from a \texttt{fort.12} file.

        Note that if the file is used as an input file for the initial distribution, the closed orbit is not added, even if requested with the \texttt{ADD\_CLORB} flag.
        \index{fort.13}\index{fort.12}
    } \\*
    \hline

    \rowcolor{gray!15}
    \texttt{WRITE\_FORT12} & \texttt{interval(int)} & \texttt{10000} \\*
    \hline
    \multicolumn{3}{|>{\raggedright}p{\textwidth}|}{%
        How often, in terms of turns, to write the particle distribution to file \texttt{fort.12}.
        This file can be renamed to \texttt{fort.13} and used as an input file for continued tracking.
        \index{fort.13}\index{fort.12}
    } \\*
    \hline

    \rowcolor{gray!15}
    \texttt{WRITE\_TRACKS} & \texttt{interval(int) [rewind(flag)]} & \texttt{nturn+1 on} \\*
    \hline
    \multicolumn{3}{|>{\raggedright}p{\textwidth}|}{%
        How often, in terms of turns, to write to the tracking files on unit 90 and lower (see Appendix~\ref{Files}).
        The optional \texttt{rewind} flag specifies whether or not to rewind the tracking files on each write.
        \index{singletrackfile}\index{trackfiles}
    } \\*
    \hline

    \rowcolor{blue!15}
    \multicolumn{3}{|c|}{\textbf{Various Flags and Options}}\\*
    \hline

    \rowcolor{gray!15}
    \texttt{EXACT} & \texttt{on|off} & \texttt{off} \\*
    \hline
    \multicolumn{3}{|>{\raggedright}p{\textwidth}|}{%
        Switch to enable exact solution of the equation of motion into tracking and 6D (no 4D) optics calculations.
        \begin{equation*}
            \mbox{\texttt{off:}}
            \quad x^\prime \simeq \frac{P_x}{P_0(1+\delta)},
            \quad y^\prime \simeq \frac{P_y}{P_0(1+\delta)};
        \end{equation*}
        \begin{equation*}
            \mbox{\texttt{on:}}
            \quad x^\prime \simeq \frac{P_x}{P_0\sqrt{(1+\delta)^2-P_x^2-P_y^2}},
            \quad y^\prime \simeq \frac{P_y}{P_0\sqrt{(1+\delta)^2-P_x^2-P_y^2}}.
        \end{equation*}
        \index{equation of motion}
    } \\*
    \hline

    \rowcolor{gray!15}
    \texttt{CURVEFF} & \texttt{on|off} & \texttt{off} \\*
    \hline
    \multicolumn{3}{|>{\raggedright}p{\textwidth}|}{%
        Enable or disable the effect of the curvature in a combined function magnet (bending + quadrupole).
        Note that the weak focusing effect is always included.
        \index{curve effect}\index{combined function magnet}
    } \\*
    \hline

\end{longtable}
\end{center}

% ================================================================================================================================ %
\section{Tracking Parameters} \label{TraPar}

All tracking parameters are defined with this input block.\index{tracking}\index{TRAC}
The initial coordinates are generally also set here.
A fine tuning of the initial condition is done with \textit{Initial Coordinates} block (\ref{IniCoo}), and the parameters for the synchrotron oscillation are given in block (\ref{SynOsc}).

\bigskip
\begin{tabular}{@{}llp{0.7\linewidth}}
    \textbf{Keyword}    & \texttt{TRAC}\index{TRAC} &\\
    \textbf{Data lines} & 3 &\\
    \textbf{Format}     & Line 1: & \texttt{numl numlr napx amp(1) amp0 ird imc} \\
                        &         & \texttt{niu(1) niu(2) numlcp numlmax} \\
                        & Line 2: & \texttt{idy(1) idy(2) idfor irew iclo6} \\
                        & Line 3: & \texttt{nde(1) nde(2) nwr(1) nwr(2) nwr(3) nwr(4)} \\
                        &         & \texttt{ntwin ibidu iexact curveff}
\end{tabular}

\paragraph{Format Description}~

\bigskip
\begin{longtabu}{@{}llp{0.7\linewidth}}
    \texttt{numl}          & integer  & Number of turns in the forward direction\index{turn}. \\
    \texttt{numlr}         & integer  & Number of turns in the backward direction\index{reverse turn}. \\
    \texttt{napx}          & integer  & Number of amplitude variations (i.e.\ particle pairs)\index{particle pairs}. \\
    \texttt{amp(1),amp0}   & floats   & Start and end amplitude (any sign) in the horizontal phase space plane for the amplitude variations. The vertical amplitude is calculated using the ratio between the horizontal and vertical emittance set in the \textit{Initial Coordinates} block (\ref{IniCoo}), where the initial phase in phase space are also set. Additional information can be found in the \textit{Remarks}. \\
    \texttt{imc}           & integer  & Number of variations of the relative momentum deviation\index{momentum deviations} has been removed. This value must be 1.\\
    \texttt{niu(1),niu(2)} & integer  & Start and stop structure element index for optics calculation. If 0, defaults to the full machine. \\
    \texttt{numlcp}        & integer  & Checkpoint/restart\index{checkpoint/restart} version: How often to write checkpointing files. \\
    \texttt{numlmax}       & integer  & No longer in use. \\
    \texttt{idz(1),idz(2)} & integers & A tracking where one of the transversal motion planes shall be ignored is only possible when all coupling terms are switched off. The part of the coupling that is due to closed orbit and other effects can be turned off with these switches. \\
                           &          & \texttt{idz(1), idz(2) = 1}: coupling on. \\
                           &          & \texttt{idz(1), idz(2) = 0}: coupling to the horizontal and vertical motion plane respectively switched off. \\
    \texttt{idfor}         & integer  & Usually the closed orbit is added to the initial coordinates. This can be turned off using \texttt{idfor}, for instance when a run is to be prolonged. \\
                           &          & \texttt{idfor = 0}: closed orbit added. \\
                           &          & \texttt{idfor = 1}: initial coordinates\index{initial coordinates} unchanged. \\
                           &          & \texttt{idfor = 2}: prolongation of a run, taken the initial coordinates from \texttt{fort.13}\index{fort.15}. \\
    \texttt{irew}          & integer  & To reduce the amount of tracking data after each amplitude and relative momentum deviation iteration $\Delta p/p_0$ the binary output units 90 and lower (see Appendix~\ref{Files}) are rewound. This is always done when the post-processing is activated (\ref{PosPro}). For certain applications it may be useful to store all data. The switch \texttt{irew} allows for that. \\
                           &          & \texttt{irew = 0}: unit 90 (and lower) rewound. \\
                           &          & \texttt{irew = 1}: all data on unit 90 (and lower). \\
    \texttt{iclo6}         & integer  & This switch allows to calculate the 6D closed orbit and optical functions at the starting point, using the differential algebra package. It is active in all versions that link to the Differential Algebra package.  Note that \texttt{iclo6 > 0} is mandatory for 6D simulations, and that \texttt{iclo6 = 0} is mandatory for 4D simulations. \\
                           &          & \texttt{iclo6 = 0}: switched off. \\
                           &          & \texttt{iclo6 = 1}: calculated. \\
                           &          & \texttt{iclo6 = 2}: calculated and added to the initial coordinates (\ref{IniCoo}). \\
                           &          & \texttt{iclo6 = 5 or 6}: like for 1 and 2, but in addition a guess closed orbit is read (in free format) from file \texttt{fort.33}. \\
    \texttt{nde(1)}        & integer  & Number of turns at flat bottom, useful for energy ramping. \\
    \texttt{nde(2)}        & integer  & Number of turns for the energy ramping. \texttt{numl}-\texttt{nde(2)} gives the number of turns on the flat top. For constant energy with \mbox{$nde(1) = nde(2) = 0$} the particles are considered to be on the flat top. \\
    \texttt{nwr(1)}        & integer  & Every \texttt{nwr(1)}'th turn the coordinates will be written on unit 90 (and lower) in the flat bottom part of the tracking. \\
    \texttt{nwr(2)}        & integer  & Every \texttt{nwr(2)}'th turn the coordinates in the ramping region will be written on unit 90 (and lower). \\
    \texttt{nwr(3)}        & integer  & Every \texttt{nwr(3)}'th turn at the flat top a write out of the coordinates on unit 90 (and lower) will occur. For constant energy this number controls the amount of data on unit 90 (and lower), as the particles are considered on the flat top. \\
    \texttt{nwr(4)}        & integer  & In cases of very long runs it is sometimes useful to save all coordinates for a prolongation of a run after a possible crash of the computer. Every \texttt{nwr(4)}'th turn the coordinates are written to unit 6. \\
    \texttt{ntwin}         & integer  & For the analysis of the Lyapunov exponent\index{Lyapunov exponent} it is usually sufficient to store the calculated distance of phase space together with the coordinate of the first particle (\texttt{ntwin} set to one). You may want to improve the 6D calculation of the distance in phase space with \texttt{sigcor, dpscor} (see~\ref{IniCoo}) when the 6D closed orbit is not calculated with \texttt{iclo6} $\neq 2$. If storage space is no problem, one can store the coordinates of both particles (\texttt{ntwin} set to two). The distance in phase space is then calculated in the post-processing procedure (see~\ref{PosPro}). This also allows a subsequent refined Lyapunov analysis using differential algebra and Lie algebra\index{Lie algebra} techniques (\cite{Refine}). \\
    \texttt{ibidu}         & integer  & No longer in use. Value ignored. \\
    \texttt{iexact}        & integer  & Switch to enable exact solution of the equation of motion into tracking and 6D (no 4D) optics calculations. \\
                           &          & \texttt{iexact = 0}: approximated equation
                           \begin{equation*}
                                \mbox{e.g.}
                                \quad x^\prime \simeq \frac{P_x}{P_0(1+\delta)},
                                \quad y^\prime \simeq \frac{P_y}{P_0(1+\delta)};
                           \end{equation*} \\
                           &          & \texttt{iexact = 1}: exact equation
                           \begin{equation*}
                                \mbox{e.g.}
                                \quad x^\prime \simeq \frac{P_x}{P_0\sqrt{(1+\delta)^2-P_x^2-P_y^2}},
                                \quad y^\prime \simeq \frac{P_y}{P_0\sqrt{(1+\delta)^2-P_x^2-P_y^2}}.
                           \end{equation*} \\
    \texttt{curveff}       & integer  & \texttt{curveff = 0}: the effect of the curvature in a combined function is neglected. Note that the weak focusing effect is always included. \\
                           &          & \texttt{curveff = 1}: switch to enable the curvature effect in a combined function magnet (bending + quadrupole).
\end{longtabu}

\paragraph{Remarks}~
\begin{enumerate}
    \item This input data block is usually combined with the \textit{Initial Coordinates}\index{INIT} input block (\ref{IniCoo}) to allow a flexible choice of the initial coordinates for the tracking.
    \item For a prolongation of a run the following parameters have to be set:
    \begin{enumerate}
        \item in this input block: \texttt{idfor = 1}
        \item in the \textit{Initial coordinates} input block:
        \begin{itemize}
            \item \texttt{itra = 0}
            \item take the end coordinates of the previous run as the initial coordinates (including all digits) for the new run.
        \end{itemize}
    \end{enumerate}
    \item A feature is installed for a prolongation of a run by using \texttt{idfor = 2} and reading the initial data from file \texttt{fort.13}. The end coordinates are now written to \texttt{fort.12} after each run. Intermediate coordinates are also written to \texttt{fort.12} in case the turn number \texttt{nwr(4)} is exceeded in the run. The user takes responsibility to transfer the required data from \texttt{fort.12} to \texttt{fort.13} if a prolongation is requested. This feature can be used to effectively read in a custom-made beam distribution. The format of the file is one line per pair of particles; the meaning of the columns is exactly that of the \textit{Initial Coordinates}\index{INIT} input block (see Sec.~\ref{IniCoo} and Tab.~\ref{T-IniCoo}).
    \item As of version 5.2 the particle momentum offset from \texttt{fort.13} is re-calculated from the particle energy to ensure these are consistent. The particle momentun offset in \texttt{fort.13} is therefore ignored.
    \item Some illogical combinations of parameters have been suppressed.
    \item The initial coordinates are calculated using a proper linear 6D transformation: \texttt{amp(1)} is still the maximum horizontal starting amplitude (excluding the dispersion contribution) from which the emittance of mode 1 $e_I$ is derived, \texttt{rat} (see~\ref{IniCoo}) is the ratio of $e_{II}/e_I$ of the emittances of the two modes. The momentum deviation $\frac{\Delta p}{p_{0,1}}$ is used to define a longitudinal amplitude. The 6 normalized coordinates read:
        \begin{enumerate}
            \item horizontal:\\
            \begin{equation*}
                \left[\sqrt{e_I} = \frac{\mbox{amp(1)}} {\sqrt{\beta_{xI}}+\sqrt{\left|\mbox{rat}\right|\times\beta_{xII}}}, \quad 0.0 \right]
            \end{equation*}
            \item vertical:\\
            \begin{equation*}
                \left[sign(\mbox{rat})\times \sqrt{e_{II}} \mbox{ with } e_{II} = \left|\mbox{rat}\right|\times e_{I}, \quad 0.0 \right]
            \end{equation*}
            \item longitudinal:\\
            \begin{equation*}
                \left[0.0, \quad \frac{\Delta p}{p_{0,1}} \times \sqrt{\beta_{sIII}} \right]
            \end{equation*}
        \end{enumerate}
        and are then transformed with the 6D linear transformation into real space. Note that results may differ from those of older versions.
    \item The amplitude scan is performed from \texttt{amp(1)} to \texttt{amp0} in steps of $\mbox{delta} = (\mbox{amp0} - \mbox{amp(1)}) / (napx-1)$. For the intermediate amplitudes, \texttt{delta} is added up for each step, however the last amplitude is guaranteed to be fixed to the given value. This enables ``control calculations'' by setting the first amplitude of one simulation equal to the last amplitude of another simulation, and unless there are calculation errors, they shall produce exactly the same results.
    \item Note that if \texttt{iclo6 = 2} and \texttt{idfor = 0} in the input file, then \texttt{idfor} is internally set to 1, as is seen in some outputs. This does not mean that the closed orbit is not added; the setting of \texttt{iclo6 = 2} simply takes precedence.
\end{enumerate}

% ================================================================================================================================ %
\section{Initial Coordinates} \label{IniCoo}

The \textit{Initial Coordinates} input block is meant to manipulate how the initial coordinates are organise, which are generally set in the tracking parameter block (\ref{TraPar}).\index{initial coordinates}\index{INIT}
Number of particles\index{particle pairs}, initial phase, ratio of the horizontal and vertical emittances and increments of \mbox{2 $\times$ 6} coordinates of the two particles, the reference energy and the starting energy for the two particles.

\bigskip
\begin{tabular}{@{}llp{0.7\linewidth}}
    \textbf{Keyword}    & \texttt{INIT}\index{INIT} \\
    \textbf{Data lines} & 16 \\
    \textbf{Format}     & Line 1: \texttt{itra chi0 chid rat iver} \\
                        & Lines 2 to 16: 15 initial coordinates as listed in Table~\ref{T-IniCoo}
\end{tabular}

\paragraph{Format Description}~

\bigskip
\begin{longtabu}{@{}llp{0.7\linewidth}}
    \texttt{itra} & integer  & Number of particles: \\
                  &          & \texttt{itra = 0}: Amplitude values of tracking parameter block (\ref{TraPar}) are ignored and coordinates of data line 2--16 are taken. \texttt{itra} is set internally to 2 for tracking with two    particles. This is necessary in case a run is to be prolonged. \\
                  &          & \texttt{itra = 1}: Tracking of one particle, twin particle ignored. \\
                  &          & \texttt{itra = 2}: Tracking the two twin particles. \\
    \texttt{chi0} & float    & Starting phase of the initial coordinate in the horizontal and vertical phase space projections. \\
    \texttt{chid} & float    & Phase difference between first and second particles. \\
    \texttt{rat}  & float    & Denotes the emittance ratio ($e_{II}/e_I$) of horizontal and vertical motion. For further information see the \emph{Remarks} of the \texttt{TRAC} input block in Section~\ref{TraPar}. \\
    \texttt{iver} & integer  & In tracking with coupling it is sometimes desired to start with zero vertical amplitude which can be painful if the emittance ratio \texttt{rat} is used to achieve it. For this purpose the switch \texttt{iver} has been introduced: \\
                  &          & \texttt{iver = 0}: Vertical coordinates unchanged. \\
                  &          & \texttt{iver = 1}: Vertical coordinates set to zero.
\end{longtabu}

\begin{table}[h]
    \caption{Initial Coordinates of the 2 Particles}
    \label{T-IniCoo}
    \centering
    \begin{tabular}{|l|l|}
        \hline
        \rowcolor{blue!30}
        Line & Contents \\
        \hline
        2 & $x_1$ [mm] coordinate of particle 1 \\
        \hline
        3 & $x'_1$ [mrad] coordinate of particle 1 \\
        \hline
        4 & $y_1$ [mm] coordinate of particle 1 \\
        \hline
        5 & $y'_1$ [mrad] coordinate of particle 1 \\
        \hline
        6 & path length difference 1 ($\sigma_1 = s - v_0 \times t$) [mm] of particle 1 \\
        \hline
        7 & $\Delta p/p_{0,1} $ of particle 1 \\
        \hline
        8 & $x_2$ [mm] coordinate of particle 2 \\
        \hline
        9 & $x'_2$ [mrad] coordinate of particle 2 \\
        \hline
        10 & $y_2$ [mm] coordinate of particle 2 \\
        \hline
        11 & $y'_2$ [mrad] coordinate of particle 2 \\
        \hline
        12 & path length difference ($ \sigma_2 = s - v_0 \times t$) [mm] of particle 2 \\
        \hline
        13 & $\Delta p/p_{0,2}$ of particle 2 \\
        \hline
        14 & energy [MeV] of the reference particle\index{reference energy} \\
        \hline
        15 & energy [MeV] of particle 1 \\
        \hline
        16 & energy [MeV] of particle 2 \\
        \hline
    \end{tabular}
\end{table}

\paragraph{Remarks}~
\begin{itemize}
    \item These 15 coordinates are taken as the initial coordinates if \texttt{itra} is set to zero (see above). If \texttt{itra} is 1 or 2 these coordinates are added to the initial coordinates generally defined in the tracking parameter block (\ref{TraPar}). This procedure seems complicated but it allows freely to define the initial difference between the two twin particles. It also allows in case a tracking run should be prolonged to continue with precisely the same coordinates. This is important as small difference may lead to largely different results.
    \item The reference particle is the particle in the centre of the bucket which performs no synchrotron oscillations.
    \item The energy of the first and second particles is given explicitly, again to make possible a continuation that leads precisely to the same results as if the run would not have been interrupted.
    \item There is a refined way of prolonging a run, see the \textit{Tracking Parameters} input block (\ref{TraPar}).
\end{itemize}

% ================================================================================================================================ %
\section{Synchrotron Oscillation} \label{SynOsc}

The parameters needed for treating the synchrotron oscillation in a symplectic manner are given in the \textit{Synchrotron Oscillation} input block.\index{synchrotron oscillation}\index{SYNC}

\bigskip
\begin{tabular}{@{}llp{0.7\linewidth}}
    \textbf{Keyword}    & \texttt{SYNC}\index{SYNC} \\
    \textbf{Data lines} & 2 \\
    \textbf{Format}     & Line 1: \texttt{harm alc u0 phag tlen pma ition dppoff} \\
                        & Line 2: \texttt{dpscor sigcor}
\end{tabular}

\paragraph{Format Description}~

\bigskip
\begin{longtabu}{@{}llp{0.65\linewidth}}
    \texttt{harm}   & integer & Harmonic number\index{harmonic number}. \\
    \texttt{alc}    & float   & Momentum compaction\index{momentum compactions} factor, used here only to calculate the linear synchrotron tune $Q_{S}$. \\
    \texttt{u0}     & float   & Circumference voltage\index{circumference voltage} in [MV]. \\
    \texttt{phag}   & float   & Acceleration phase\index{acceleration phase} in degrees. \\
    \texttt{tlen}   & float   & Length of the accelerator\index{accelerator length} in meters. \\
    \texttt{pma}    & float   & Rest mass\index{rest mass} of the particle in $\mathrm{MeV}/\mathrm{c}^{2}$. \\
    \texttt{ition}  & integer & Transition energy switch\index{transition energy}: \\
                    &         & \texttt{ition = 0}: for no synchrotron oscillation (energy ramping still possible). \\
                    &         & \texttt{ition = 1}: for above transition energy. \\
                    &         & \texttt{ition = -1}: for below transition energy. \\
    \texttt{dppoff} & float   & Offset Relative Momentum Deviation\index{relative momentum deviation} $\Delta p/p_0$: a fixpoint with respect to synchrotron oscillations. It becomes active when the 6D closed orbit is calculated (see item \texttt{iclo6} in section~\ref{TraPar}). \\
    \texttt{dpscor,sigcor} & floats & Scaling factor for relative momentum deviation $\Delta p/p_0$ and the path length difference ($\sigma = s - v_0 \times t$) respectively. They can be used to improve the calculation of the 6D distance in phase space, but is only used when \texttt{ntwin = 1} in the tracking parameter input block~(\ref{TraPar}). Please set to 1 when the 6D closed is calculated.
\end{longtabu}

\textbf{Note:} The value of \texttt{tlen} is also calculated internally by SixTrack (in \texttt{dcum}\index{dcum}), and a warning is issued if the given value is different from the calculated value.

% ================================================================================================================================ %
\section{Tracking with Ions} \label{hions}

The default tracking in SixTrack is for protons.
In case tracking of ions is wanted the following input block should be used.\index{ion tracking}
The \texttt{HION} block only specifies the reference particle.
By default, all particles are initialised to the same values, but if multiple ion species are needed, these can be provided by an input file in the \texttt{DIST} block.

\bigskip
\begin{tabular}{@{}llp{0.7\linewidth}}
    \textbf{Keyword}    & \texttt{HION}\index{HION} \\
    \textbf{Data lines} & 1 \\
    \textbf{Format}     & Line 1: \texttt{A Z $m_a$ Q}
\end{tabular}

\paragraph{Format Description}~

\bigskip
\begin{longtabu}{@{}llp{0.65\linewidth}}
    \texttt{A}     & integer & Total number of nucleons (atomic mass number). \\
    \texttt{Z}     & integer & Total number of protons. \\
    \texttt{$m_a$} & float   & Mass of the ion [GeV/$c^2$]. \\
    \texttt{Q}     & integer & Electrical charge.
\end{longtabu}

% ================================================================================================================================ %
\section{Initial Distribution} \label{distBlock}

The \texttt{DIST} block adds the ability to read a beam distribution from file, or generate it internally in SixTrack.\index{DIST}\index{initial distribution}
The file format is very flexible and can be specified column-wise with the \texttt{FORMAT} keyword to support many file layouts.
It is also possible to specify the unit of the data in the input file, within a limited range.
If no format is specified, the \texttt{DIST} block falls back to the fixed 14 column format read by the original simple \texttt{DIST} block prior to version 5.3.1 (see Table~\ref{tab:distReadFileColumns}).

\bigskip
\begin{tabular}{@{}lp{0.7\linewidth}}
    \textbf{Keyword}    & \texttt{DIST}\index{DIST} \\
    \textbf{Data lines} & Variable \\
    \textbf{Format}     & Keyword/value
\end{tabular}

\paragraph{Format Description}~\\

There are several approaches available for initiating the beam distribution.
It can be read in its entirety from a file, selecting a set of many available conventions for describing the particle coordinates, ion values and meta data.
For a small number of particles, the coordinates can be set directly in the \texttt{DIST} block as well.

The \texttt{DIST} block is linked to the external \texttt{DISTlib} library for beam distributions, and the filled particle coordinates can be passed on to this library for further processing, like applying the T-matrix (see Section~\ref{Sec:TMatrix}).
In the future, more features will be added to this library, and made available through the \texttt{DIST} block interface.

Table~\ref{Table:DIST} lists the currently available keywords of the \texttt{DIST} block.
Table~\ref{Table:DIST_FORMAT} lists all the column formats the block supports.
Table~\ref{Table:DIST_FILL} lists all the available fill methods for populating the particle arrays without having to read from file.

For an overview of the definitions used for the particle tracking variables in SixTrack, see Section~\ref{Sec:PartArrays}.

\paragraph{Default Behaviour}~\\

Since there are multiple ways to set the particle coordinates, a few default behaviours and precedences have been coded into the parsing.

\begin{itemize}
    \item All particle coordinates are initially set to 0, with the exception of particle energy, which defaults to the reference energy. That is, the particle $\delta$ momentum is 0.
    \item The particle mass and ion parameters also default to those of the reference particle.
    \item The particle ID, if not provided, is set as a range from 1 to the number of particles as specified in the \texttt{TRAC} or \texttt{SIMU} block.
    \item The particle parent ID is set to equal that of its ID. This is the correct way to indicate that a particle is a primary particle.
\end{itemize}

It is only possible to set each particle coordinate using one method.
If conflicting methods are selected in the \texttt{FORMAT} keyword and in \texttt{FILL} methods, an error will be raised.
Since SixTrack uses multiple arrays for different values related to the particle energy, these are calculated after initialisation from the input format chosen.

The default normalisation method is to use the internal T-matrix in SixTrack (see Section~\ref{Sec:TMatrix}).
Alternatively, Twiss and dispersion can be set in the block, or a new T-matrix provided.
These will then take precedence over the internal matrix.

All emittances default to 0, so if these are not set, the normalisation will also return arrays of zeros.

\begin{center}
\setlength\LTleft{0pt}
\setlength\LTright{0pt}
\begin{longtable}{@{\extracolsep{\fill}}|l|p{10cm}|l|}
    \caption{Available keyword/value sets in the DIST block.}
    \label{Table:DIST} \\*
    \hline
    \rowcolor{blue!30}
    \textbf{Keyword} & \textbf{Argument(s)} & \textbf{Default} \\*
    \hline
    \endfirsthead

    \hline
    \rowcolor{blue!30}
    \textbf{Keyword} & \textbf{Argument(s)} & \textbf{Default} \\*
    \endhead

    \rowcolor{gray!15}
    \multicolumn{3}{|c|}{(The table continues on the next page)}\\*
    \hline
    \endfoot

    \hline
    \endlastfoot

    \rowcolor{blue!15}
    \multicolumn{3}{|c|}{\textbf{Input, Output and Format}}\\*
    \hline

    \rowcolor{gray!15}
    \texttt{FORMAT} & \texttt{[list\_of\_columns]} & \texttt{OLD\_DIST}\\*
    \hline
    \multicolumn{3}{|>{\raggedright}p{\textwidth}|}{%
        A list of column formats for the input. The available column values are listed in Table~\ref{Table:DIST_FORMAT}. This format is applied to either the input file or to particles specified directly in the \texttt{DIST} block. The number format columns must match the file columns or particle entries. If no format is specified, the parser assumes it will receive a 14 column file matching the format described in Table~\ref{tab:distReadFileColumns}.
    } \\*
    \hline

    \rowcolor{gray!15}
    \texttt{READ} & \texttt{filename(char) [use\_distlib(flag)]} & \\*
    \hline
    \multicolumn{3}{|>{\raggedright}p{\textwidth}|}{%
        The filename of the file to read. An optional logical flag sets whether the filename is passed on to the external DISTlib, in which case the file must conform to the DISTlib file format. This is not covered here. If the file contains more particles than requested in the \texttt{TRAC} or \texttt{SIMU} block, the remaining particles will be ignored. If the file contains less particles, an error will be raised.
    } \\*
    \hline

    \rowcolor{gray!15}
    \texttt{PARTICLE} & \texttt{[list\_of\_values]} & \\*
    \hline
    \multicolumn{3}{|>{\raggedright}p{\textwidth}|}{%
        A list of values to be parsed as a particle. This requires a format to be specified. It provides the option to add particles to the simulation without having to use the \texttt{INIT} block or a distribution file. Although not intended for initialising a large number of particles, there is no limit on how many times this keyword can be used.
    } \\*
    \hline

    \rowcolor{gray!15}
    \texttt{ECHO} & \texttt{[filename]} & \texttt{echo\_distribution.dat}\\*
    \hline
    \multicolumn{3}{|>{\raggedright}p{\textwidth}|}{%
        Echos the distribution back to a file. The format of the file is described in Table~\ref{tab:distEchoFileColumns}. This keyword is kept for legacy support, but a much more detailed file is written by the \texttt{INITIALSTATE} keyword in the \texttt{SETTINGS} block.
    } \\*
    \hline

    \rowcolor{blue!15}
    \multicolumn{3}{|c|}{\textbf{Beam Parameters}}\\*
    \hline

    \rowcolor{gray!15}
    \texttt{EMITTANCE} & \texttt{emit1[mm mrad] emit2[mm mrad]} & \texttt{0.0 0.0}\\*
    \hline
    \multicolumn{3}{|>{\raggedright}p{\textwidth}|}{%
        The transverse beam emittance values in units of mm mrad.
    } \\*
    \hline

    \rowcolor{gray!15}
    \texttt{LONGEMIT} & \texttt{emit3 unit[eVs|um]} & \texttt{0.0}\\*
    \hline
    \multicolumn{3}{|>{\raggedright}p{\textwidth}|}{%
        Longitudinal emittance and its unit. The emittance can either be provided in $\mu$m or in eVs.
    } \\*
    \hline

    \rowcolor{gray!15}
    \texttt{TWISS} & \texttt{betaX[m] alphaX[1] betaY[m] alphaY[1]} & \texttt{1.0 0.0 1.0 0.0}\\*
    \hline
    \multicolumn{3}{|>{\raggedright}p{\textwidth}|}{%
        The horizontal and vertical twiss parameters.
    } \\*
    \hline

    \rowcolor{gray!15}
    \texttt{DISPERSION} & \texttt{dx dpx dy dpy} & \texttt{0.0 0.0 0.0 0.0}\\*
    \hline
    \multicolumn{3}{|>{\raggedright}p{\textwidth}|}{%
        Beam dispersion
    } \\*
    \hline

    \rowcolor{gray!15}
    \texttt{TMATRIX} & \texttt{val1 val2 val3 val4 val5 val6} & \\*
    \hline
    \multicolumn{3}{|>{\raggedright}p{\textwidth}|}{%
        Specify the $6 \times 6$ normalisation matrix in its entirety. The keyword needs to be repeated 6 times, once for each row.
    } \\*
    \hline

    \rowcolor{blue!15}
    \multicolumn{3}{|c|}{\textbf{Internal Generator}}\\*
    \hline

    \rowcolor{gray!15}
    \texttt{FILL} & \texttt{[list\_of\_parameters]} & \\*
    \hline
    \multicolumn{3}{|>{\raggedright}p{\textwidth}|}{%
        The different columns can also be filled by a set of fill functions controlled with this keyword. The settings provided by this keyword are applied after the file is read or the arrays are populated by the \texttt{PARTICLES} keyword. The \texttt{FILL} feature can therefore be used to overwrite the data read from the file. The various fill methods available are listed in Table~\ref{Table:DIST_FILL}.
    } \\*
    \hline

    \rowcolor{gray!15}
    \texttt{SEED} & \texttt{seed(int)} & \\*
    \hline
    \multicolumn{3}{|>{\raggedright}p{\textwidth}|}{%
        Provide a seed for the random number generator. This seed will only be used by the \texttt{DIST} block, and does not affect random numbers generated in other parts of SixTrack. If one of the \texttt{FILL} keywords uses a random number generator, a seed must be provided.
    } \\*
    \hline

\end{longtable}
\end{center}

\subsection{Column Formats}

The \texttt{FORMAT} keyword allows the user to specify their own file format by providing a list of columns it contains.
The available formats and how they are converted to internal SixTrack particle coordinates are listed in Table~\ref{Table:DIST_FORMAT}.

Each column can optionally take a unit in square brackets, appended to the column name itself with no space in between.
The available units are also listed in the table for the column formats that support units.
If no unit in square brackets is provided, the parser defaults to the internal SixTrack units which are mm, mrad and MeV.

For units of energy $c=1$ such that for instance MeV, MeV/c and MeV/c$^2$ are equivalent.
The parser accepts the following notation: \texttt{MeV}, \texttt{MeV/c}, \texttt{MeV/c\^{}2}, and \texttt{MeV/c**2}.
Units are not case sensitive.

For units of length, the parser accepts the character \texttt{u} as an alternative to $\mu$.

\paragraph{Multi-Column Formats}~\\

To avoid the need for specifiying common combinations of columns, a set of multi-column keywords are also available.
They are translated directly into a group of columns in a pre-defined order.
using these keywords does not prevent the user from adding more columns to extend the format.

Note, however, that conflicting columns cannot be provided.
Only one column for each of the 6 particle coordinates is allowed at the same time.
If the file contains multiple columns for the same coordinate, the columns not in use can be disabled with the \texttt{SKIP} flag.

For backwards compatibility with the old \texttt{DIST} block, a format that matches the old 14 colum file is also provided.
For the recent addition of charge and PDGID, these columns must be added to this format.
If no format is specified, the default is the old 14 column format described in Table~\ref{tab:distReadFileColumns}.

\paragraph{Example}~\\

Below is an example of a \texttt{DIST} block using a 7-column input file with the length unit in millimetres.

\begin{cverbatim}
DIST
  FORMAT ID X[mm] PX Y[mm] PY ZETA[mm] DELTA
  READ   partDist.dat
NEXT
\end{cverbatim}

\begin{center}
\setlength\LTleft{0pt}
\setlength\LTright{0pt}
\begin{longtable}{@{\extracolsep{\fill}}|p{10cm}|l|}
    \caption{Available column formats in the DIST block.}
    \label{Table:DIST_FORMAT} \\*
    \hline
    \rowcolor{blue!30}
    \textbf{Column Name} & \textbf{Units} \\*
    \hline
    \endfirsthead

    \hline
    \rowcolor{blue!30}
    \textbf{Column Name} & \textbf{Units} \\*
    \endhead

    \rowcolor{gray!15}
    \multicolumn{2}{|c|}{(The table continues on the next page)}\\*
    \hline
    \endfoot

    \hline
    \endlastfoot

    \rowcolor{blue!15}
    \multicolumn{2}{|c|}{\textbf{Meta Columns}}\\*
    \hline

    \rowcolor{gray!15}
    \texttt{SKIP} & N/A\\*
    \hline
    \multicolumn{2}{|>{\raggedright}p{\textwidth}|}{%
        Disables the column in the file, that is, during parsing, the column is skipped.
    } \\*
    \hline

    \rowcolor{gray!15}
    \texttt{ID} & N/A\\*
    \hline
    \multicolumn{2}{|>{\raggedright}p{\textwidth}|}{%
        The particle ID. Currently, this number must be in the range 1 to number of particles in the simulation, and they must be unique. There is no restriction on the order.
    } \\*
    \hline

    \rowcolor{gray!15}
    \texttt{PARENT} & N/A\\*
    \hline
    \multicolumn{2}{|>{\raggedright}p{\textwidth}|}{%
        The particle's parent ID. If the parent ID is the same as the particle ID, the particle is considerd a primary particle.
    } \\*
    \hline

    \rowcolor{blue!15}
    \multicolumn{2}{|c|}{\textbf{Transverse Coordinates}}\\*
    \hline

    \rowcolor{gray!15}
    \texttt{X, Y} & m or mm\\*
    \hline
    \multicolumn{2}{|>{\raggedright}p{\textwidth}|}{%
        The particle coordinate in the horizontal and vertical plane, respectively. These are the internal values used for tracking in SixTrack, and are read in as provides.
    } \\*
    \hline

    \rowcolor{gray!15}
    \texttt{XP, YP} & [1] \\*
    \hline
    \multicolumn{2}{|>{\raggedright}p{\textwidth}|}{%
        The particle transverse momentum ratio relative to its total momentum, $p_x/p \approx x^\prime$, $p_y/p \approx y^\prime$. These are the internal values used for tracking in SixTrack, and are read in as provides.
    } \\*
    \hline

    \rowcolor{gray!15}
    \texttt{PX, PY} & eV, keV, MeV, GeV or TeV \\*
    \hline
    \multicolumn{2}{|>{\raggedright}p{\textwidth}|}{%
        The particle transverse momentum. These values will be converted to SixTrack internal values.
    } \\*
    \hline

    \rowcolor{gray!15}
    \texttt{PX/P0, PXP0, PY/P0, PYP0} & [1] \\*
    \hline
    \multicolumn{2}{|>{\raggedright}p{\textwidth}|}{%
        The particle transverse momentum relative to the reference momentum, $p_x/p_0$, $p_y/p_0$. The slash in the column name is optional. These values will be converted to SixTrack internal values.
    } \\*
    \hline

    \rowcolor{blue!15}
    \multicolumn{2}{|c|}{\textbf{Longitudinal Position}}\\*
    \hline

    \rowcolor{gray!15}
    \texttt{SIGMA} & m or mm \\*
    \hline
    \multicolumn{2}{|>{\raggedright}p{\textwidth}|}{%
        The particle offset relative to the reference particle, $\sigma$. This is the internal value used for tracking in SixTrack, and is read in as provided.
    } \\*
    \hline

    \rowcolor{gray!15}
    \texttt{ZETA} & m or mm \\*
    \hline
    \multicolumn{2}{|>{\raggedright}p{\textwidth}|}{%
        The particle offset relative to the reference particle, with relative velocity correction, $\zeta = \frac{\beta_0}{\beta}\sigma$. This value will be converted to SixTrack internal value \texttt{SIGMA} ($\sigma$).
    } \\*
    \hline

    \rowcolor{gray!15}
    \texttt{DT} & ps, ns, $\mu$s, ms or s \\*
    \hline
    \multicolumn{2}{|>{\raggedright}p{\textwidth}|}{%
        The particle time delay, $\sigma = -\beta_0 \cdot \mathrm{d}t \cdot c$. This value will be converted to SixTrack internal value.
    } \\*
    \hline

    \rowcolor{blue!15}
    \multicolumn{2}{|c|}{\textbf{Energy and Momentum}}\\*
    \hline

    \rowcolor{gray!15}
    \texttt{E, P} & eV, keV, MeV, GeV or TeV \\*
    \hline
    \multicolumn{2}{|>{\raggedright}p{\textwidth}|}{%
        The particle total energy or momentum, respectively. These are values used by SixTrack for tracking. Only one can be set, and the other is computed from the first.
    } \\*
    \hline

    \rowcolor{gray!15}
    \texttt{DE/E0, DEE0} & [1] \\*
    \hline
    \multicolumn{2}{|>{\raggedright}p{\textwidth}|}{%
        The particle total energy relative to reference energy, $\Delta E / E_0$. This is converted to SixTrack internal value $E$ after input.
    } \\*
    \hline

    \rowcolor{gray!15}
    \texttt{DP/P0, DPP0, DELTA} & [1] \\*
    \hline
    \multicolumn{2}{|>{\raggedright}p{\textwidth}|}{%
        The particle total momentum relative to reference momentum, $\Delta P / P_0$. This is a value used by SixTrack for tracking. If provided as input, particle total energy and momentum is calculated from this value.
    } \\*
    \hline

    \rowcolor{gray!15}
    \texttt{PT} & [1] \\*
    \hline
    \multicolumn{2}{|>{\raggedright}p{\textwidth}|}{%
        The particle total momentum relative to reference energy, $\Delta E / P_0 c$. This is converted to SixTrack internal value $E$ after input.
    } \\*
    \hline

    \rowcolor{gray!15}
    \texttt{PSIGMA} & [1] \\*
    \hline
    \multicolumn{2}{|>{\raggedright}p{\textwidth}|}{%
        The particle total momentum relative to reference energy and relativistic velocity, $\Delta E / \beta_0 P_0 c$. This is converted to SixTrack internal value $E$ after input.
    } \\*
    \hline

    \rowcolor{blue!15}
    \multicolumn{2}{|c|}{\textbf{Normalised Coordinates}}\\*
    \hline

    \rowcolor{gray!15}
    \texttt{XN, YN, ZN, PXN, PYN, PZN} & N/A \\*
    \hline
    \multicolumn{2}{|>{\raggedright}p{\textwidth}|}{%
        The six coordinates in nurmalised coordinates in units of $\sqrt{m}$. These are transformed by the beam parameters given in the \texttt{DIST} block after input.
    } \\*
    \hline

    \rowcolor{gray!15}
    \texttt{JX, JY, JZ, PHIX, PHIY, PHIZ} & N/A \\*
    \hline
    \multicolumn{2}{|>{\raggedright}p{\textwidth}|}{%
        NOT YET IMPLEMENTED!
        The six coordinates in action coordinates. These are transformed by the beam parameters given in the \texttt{DIST} block after input.
    } \\*
    \hline

    \rowcolor{blue!15}
    \multicolumn{2}{|c|}{\textbf{Ion Parameters}}\\*
    \hline

    \rowcolor{gray!15}
    \texttt{MASS, M} & eV, keV, MeV, GeV or TeV \\*
    \hline
    \multicolumn{2}{|>{\raggedright}p{\textwidth}|}{%
        The mass of the particle. The default value is the reference particle mass set in the \texttt{SIMU}, \texttt{SYNC} or \texttt{HION} block.
    } \\*
    \hline

    \rowcolor{gray!15}
    \texttt{CHARGE, Q} & N/A \\*
    \hline
    \multicolumn{2}{|>{\raggedright}p{\textwidth}|}{%
        The charge of the particle in units of elementary charge. The default value is the reference particle mass set in the \texttt{SIMU} or \texttt{HION} block.
    } \\*
    \hline

    \rowcolor{gray!15}
    \texttt{ION\_A, ION\_Z} & N/A \\*
    \hline
    \multicolumn{2}{|>{\raggedright}p{\textwidth}|}{%
        The ion A and Z values (atomic mass and atomic charge). The default value is the reference particle mass set in the \texttt{SIMU} or \texttt{HION} block. Both columns must be provided if one of them is.
    } \\*
    \hline

    \rowcolor{gray!15}
    \texttt{PDGID} & N/A \\*
    \hline
    \multicolumn{2}{|>{\raggedright}p{\textwidth}|}{%
        The Particle Data Group ID of the particle. If it is not provided, it is either calculated from A and Z if given, or set to that of the reference particle in the \texttt{SIMU} or \texttt{HION} block.
    } \\*
    \hline

    \rowcolor{blue!15}
    \multicolumn{2}{|c|}{\textbf{Spin Vector}}\\*
    \hline

    \rowcolor{gray!15}
    \texttt{SX, SY, SZ} & N/A \\*
    \hline
    \multicolumn{2}{|>{\raggedright}p{\textwidth}|}{%
        The three components of the particle spin vector. This information is currently not used for tracking, but is available for future additions to SixTrack.
    } \\*
    \hline

    \rowcolor{blue!15}
    \multicolumn{2}{|c|}{\textbf{Multi-Columns Keywords}}\\*
    \hline

    \rowcolor{gray!15}
    \texttt{4D} & N/A \\*
    \hline
    \multicolumn{2}{|>{\raggedright}p{\textwidth}|}{%
        Equivalent to setting \texttt{X PX Y PY} with default units.
    } \\*
    \hline

    \rowcolor{gray!15}
    \texttt{6D} & N/A \\*
    \hline
    \multicolumn{2}{|>{\raggedright}p{\textwidth}|}{%
        Equivalent to setting \texttt{X PX Y PY ZETA DELTA} with default units.
    } \\*
    \hline

    \rowcolor{gray!15}
    \texttt{NORM} & N/A \\*
    \hline
    \multicolumn{2}{|>{\raggedright}p{\textwidth}|}{%
        Equivalent to setting \texttt{XN PXN YN PYN ZN PZN}.
    } \\*
    \hline

    \rowcolor{gray!15}
    \texttt{ACTION} & N/A \\*
    \hline
    \multicolumn{2}{|>{\raggedright}p{\textwidth}|}{%
        Equivalent to setting \texttt{JX PHIX JY PHIY JZ PHIZ}.
    } \\*
    \hline

    \rowcolor{gray!15}
    \texttt{IONS} & N/A \\*
    \hline
    \multicolumn{2}{|>{\raggedright}p{\textwidth}|}{%
        Equivalent to setting \texttt{MASS CHARGE ION\_A ION\_Z PDGID} with mass in units GeV.
    } \\*
    \hline

    \rowcolor{gray!15}
    \texttt{SPIN} & N/A \\*
    \hline
    \multicolumn{2}{|>{\raggedright}p{\textwidth}|}{%
        Equivalent to setting \texttt{SX SY SZ}.
    } \\*
    \hline

    \rowcolor{gray!15}
    \texttt{OLD\_DIST} & N/A \\*
    \hline
    \multicolumn{2}{|>{\raggedright}p{\textwidth}|}{%
        Gives the old file format as described in Table~\ref{tab:distReadFileColumns}. That is, it's equivalent to \texttt{ID PARENT SKIP X[M] Y[M] SKIP XP[RAD] YP[RAD] SKIP ION\_A ION\_Z MASS[GEV] P[GEV] DT}.
    } \\*
    \hline

\end{longtable}
\end{center}

\subsection{Filling the Columns}

The \texttt{FILL} keyword allows the user to fill the particle arrays with values generated based on a set of parameters.
These can be both pseudo-random distributions, value ranges, or fixed values.

The fills are processed \emph{after} particles are read from file or from direct \texttt{PARTICLE} declarations in the \texttt{DIST} block.
This makes it possible to overwrite certain values after reading from file.
The \texttt{FILL} feature can also be used to populate the arrays from scratch, which can be useful for filling the normalised arrays (see example below).

The different fill methods are listed in Table~\ref{Table:DIST_FILL}.

Note that the \texttt{PARENT} column cannot be filled.
The \texttt{SPIN} columns are also currently disbaled with this feature.

Also note that the \texttt{FILL} methods cannot create corrolated distributions.
This can be achieved by using the normalised column formats, which triggers a normalisation after they have been filled.

\bigskip
\noindent The general format for this feature is:
\begin{cverbatim}
FILL column method param1 ... paramN [firstIDX lastIDX]
\end{cverbatim}

\begin{tabular}{@{}lp{0.8\linewidth}}
    \texttt{FILL}     & The keyword selecting this feature. \\
    \texttt{column}   & The target column format. Must be one of the columns described in Table~\ref{Table:DIST_FORMAT}. \\
    \texttt{method}   & The fill method. Must be one of the methods described in Table~\ref{Table:DIST_FILL}. \\
    \texttt{params}   & The fill method parameters. These vary from method to method. See Table~\ref{Table:DIST_FILL}. \\
    \texttt{firstIDX} & Optional: The first paricle index for this fill method. Defaults to particle 1. \\
    \texttt{lastIDX}  & Optional: The last paricle index for this fill method, where -1 indicates the last particle as request in the \texttt{SIMU} or \texttt{TRAC} block. Defaults to particle -1.
\end{tabular}

\paragraph{Example}~\\

The example below fills the normalised coordinate arrays with Gaussian distributions with a sigma cut, and fills the particle ID and ion columns.
Writing to the normalised coordinate arrays in this manner triggers the normalisation routine to be called, here defaulting to use the internal T-matrix since no \texttt{TWISS} or \texttt{TMATRIX} keywords are set.

\begin{cverbatim}
DIST
  EMITTANCE 2.5 2.5
  SEED 12
  FILL ID     COUNT  1   1
  FILL XN     GAUSS  1.0 0.0 5.0
  FILL PXN    GAUSS  1.0 0.0 5.0
  FILL YN     GAUSS  1.0 0.0 5.0
  FILL PYN    GAUSS  1.0 0.0 5.0
  FILL ZN     GAUSS  0.8 0.0 3.0
  FILL PZN    GAUSS  0.5 0.0 3.0
  FILL ION_A  INT    1
  FILL ION_Z  INT    1
  FILL CHARGE INT    1
  FILL MASS   FLOAT  938.272046
NEXT
\end{cverbatim}

\begin{center}
\setlength\LTleft{0pt}
\setlength\LTright{0pt}
\begin{longtable}{@{\extracolsep{\fill}}|p{3cm}|l|}
    \caption{Available fill methods in the DIST block.}
    \label{Table:DIST_FILL} \\*
    \hline
    \rowcolor{blue!30}
    \textbf{Method} & \textbf{Argumnent(s)} \\*
    \hline
    \endfirsthead

    \hline
    \rowcolor{blue!30}
    \textbf{Method} & \textbf{Argumnent(s)} \\*
    \endhead

    \rowcolor{gray!15}
    \multicolumn{2}{|c|}{(The table continues on the next page)}\\*
    \hline
    \endfoot

    \hline
    \endlastfoot

    \rowcolor{gray!15}
    \texttt{INT} & \texttt{value [first last]} \\*
    \hline
    \multicolumn{2}{|>{\raggedright}p{\textwidth}|}{%
        Sets all values to a fixed integer.
        Can be used with column formats \texttt{ION\_A}, \texttt{ION\_Z}, \texttt{CHARGE} and \texttt{PDGID}.
    } \\*
    \hline

    \rowcolor{gray!15}
    \texttt{FLOAT} & \texttt{value [first last]} \\*
    \hline
    \multicolumn{2}{|>{\raggedright}p{\textwidth}|}{%
        Sets all values to a fixed floating point value.
        Can be used with all floating point column formats.
    } \\*
    \hline

    \rowcolor{gray!15}
    \texttt{GAUSS} & \texttt{sigma mu [cut] [first last]} \\*
    \hline
    \multicolumn{2}{|>{\raggedright}p{\textwidth}|}{%
        Generates a normal random distribution with width \texttt{sigma} and offset \texttt{mu}, with an optional sigma \texttt{cut}.
        Can be used with all floating point column formats except \texttt{MASS}.
    } \\*
    \hline

    \rowcolor{gray!15}
    \texttt{UNIFORM} & \texttt{lower upper [first last]} \\*
    \hline
    \multicolumn{2}{|>{\raggedright}p{\textwidth}|}{%
        Generates a uniform random distribution between the values \texttt{lower} and \texttt{upper}.
        Can be used with all floating point column formats except \texttt{MASS}.
    } \\*
    \hline

    \rowcolor{gray!15}
    \texttt{LINEAR} & \texttt{lower upper [first last]} \\*
    \hline
    \multicolumn{2}{|>{\raggedright}p{\textwidth}|}{%
        Fills the array with floating point values ranging between the values \texttt{lower} and \texttt{upper} in equal steps.
        Can be used with all floating point column formats except \texttt{MASS}.
    } \\*
    \hline

    \rowcolor{gray!15}
    \texttt{COUNT} & \texttt{start step [first last]} \\*
    \hline
    \multicolumn{2}{|>{\raggedright}p{\textwidth}|}{%
        Fills the array with integer values starting from \texttt{start}, with a given \texttt{step}.
        Can be used with column format \texttt{ID}.
    } \\*
    \hline

\end{longtable}
\end{center}

\subsection{Support for the Old DIST Format}

Tables~\ref{tab:distReadFileColumns} and~\ref{tab:distEchoFileColumns} describe the old input file and echo file for the \texttt{DIST} block.
These formats are still supported for backwards compatibility.

\begin{center}
\begin{longtable}{|l|l|}%{|p{2.0cm} | p{1.2cm} | p{1.0cm} | p{9.0cm}|}
    \caption{Format of the ASCII file containing the distribution to be read by the \texttt{DIST} block.}
    \label{tab:distReadFileColumns} \\*
    \hline
    \rowcolor{blue!30}
    \# & Description \\*
    \hline
    \endfirsthead
    1  & particle id \\*
    2  & parent particle id \\*
    3  & statistical weight (unused) \\*
    4  & $x$ [m] \\*
    5  & $y$ [m] \\*
    6  & $z$ (unused) \\*
    7  & $x'$ [1e-3] \\*
    8  & $y'$ [1e-3] \\*
    9  & $z'$ (unused) \\*
    10 & mass number \\*
    11 & atomic number \\*
    12 & mass [GeV/c$^2$] \\*
    13 & linear momentum [GeV/c] \\*
    14 & time lag [s] \\*
    \hline
\end{longtable}
\end{center}
\begin{center}
\begin{longtable}{|l|l|l|}
    \caption{The format of the ASCII file where the distribution read by the \texttt{DIST} block is echoed. See also Table~\ref{tab:sixtrackunits} in \texttt{DUMP} for a more detailed description of the variables.}
    \label{tab:distEchoFileColumns} \\*
    \hline
    \rowcolor{blue!30}
    \# & Variable & Unit\\*
    \hline
    \endfirsthead
    1  & \texttt{xv1}   & [mm]    \\*
    2  & \texttt{yv1}   & [1e-3]  \\*
    3  & \texttt{xv2}   & [mm]    \\*
    4  & \texttt{yv2}   & [1e-3]  \\*
    5  & \texttt{sigmv} & [mm]    \\*
    6  & \texttt{ejfv}  & [MeV/c] \\*
    \hline
\end{longtable}
\end{center}

